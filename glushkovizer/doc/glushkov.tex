\section{Les automates de Glushkov}\label{sec:glushkov}

Le terme \og{}automate de Glushkov\fg{} est un abus de langage, faisant
référence aux automates que l'algorithme de transformation d'expression
régulière en automate, appelé algorithme de Glushkov produit. Son nom vient de
l'informaticien soviétique \textit{Victor Glushkov} qui est son créateur.

\subsection{Définition}

Nous appliquerons cet algorithme à l'aide de la fonction \(glushkov\) qui a
donc pour signature~:

\[
    glushkov: Exp(\Sigma) \to AFN(\Sigma, \mathbb{N})
\]

Et a ainsi, on peut définir cette fonction de cette façon~:

\begin{align*}
    glushkov (E) & = (Q, \{0\}, F, \delta) \quad \text{avec}                                                                      \\
    Q & \leftarrow \{n ~|~ n \in \mathbb{N} \land 0 \leq n < m\}                                                       \\
    F & \leftarrow \{n ~|~ (a, n) \in Last\} \cup (\{0\} \cdot Null)                                                   \\
    \forall q                    \in \delta(p, a) & ~|~ \begin{cases} (a, q) \in First, & \text{si } p = 0 \\ (a, q) \in Follow((b, p)) & \text{sinon} \end{cases} \\
    (E', m)                                       & \leftarrow linearization(E)                                                                                    \\
    (First, Last, Null, Follow)                   & \leftarrow flnl(E')
\end{align*}

\begin{example}
    Vu qu'un dessin vaut toujours mieux que mile mots, voici un exemple de
    l'automate résultant de la transformation de cette expression \(E = (a+b)
    \cdot a^* \cdot b^* \cdot (a+b)^*\).

    \begin{figure}[H]
        \centering
        \captionsetup{type=figure,justification=centering}
        \begin{tikzpicture}
            \tikzset{
                ->,
                >=stealth',
                node distance=2.25cm,
                every state/.style={thick},
                initial text=\( \),
            }
            \node[state, initial] (0) {\(0\)};
            \node[state, accepting, right of=0, below of=0] (1) {\(1\)};
            \node[state, accepting, right of=0, above of=0] (2) {\(2\)};
            \node[state, accepting, right of=2, below of=2] (3) {\(3\)};
            \node[state, accepting, right of=3] (4) {\(4\)};
            \node[state, accepting, right of=4, above of=4] (5) {\(5\)};
            \node[state, accepting, right of=5] (6) {\(6\)};

            \draw   (0) edge[below] node{\(a\)} (1)
            (0) edge[above] node{\(b\)} (2)
            (1) edge[below] node{\(a\)} (3)
            (2) edge[above] node{\(a\)} (3)
            (1) edge[bend right=1.5cm, below] node{\(b\)} (6)
            (2) edge[bend left=1.5cm, above] node{\(b\)} (6)
            (1) edge[bend right, below] node{\(a\)} (5)
            (2) edge[bend left, above] node{\(a\)} (5)
            (1) edge[bend left=2mm, below] node{\(b\)} (4)
            (2) edge[bend right=2mm, above] node{\(b\)} (4)
            (3) edge[above] node{\(b\)} (4)
            (3) edge[loop left] node{\(a\)} (3)
            (3) edge[bend left, above] node{\(a\)} (5)
            (3) edge[bend right=1.75cm, below] node{\(b\)} (6)
            (4) edge[above] node{\(a\)} (5)
            (4) edge[bend right, below] node{\(b\)} (6)
            (4) edge[loop above] node{\(b\)} (4)
            (5) edge[bend right, below] node{\(b\)} (6)
            (5) edge[loop above] node{\(a\)} (5)
            (6) edge[bend right, above] node{\(a\)} (5)
            (6) edge[loop right] node{\(b\)} (6);
        \end{tikzpicture}
        \caption{
            Exemple de représentation graphique de l'automate résultant de
            \(glushkov(E)\).
        }\label{fig:automata_glushkov}
    \end{figure}
\end{example}

\vphantom{}

Comme on peut voir sur la Figure~\ref{fig:automata_glushkov} les automates
produits sont bien souvent gros et peuvent être difficiles à comprendre, mais
une machine peut gérer ça très simplement. Outre sa taille, on peut remarquer
qu'il y a des propriétés intéressantes sur cet automate. C'est ce que l'on va
étudier maintenant.

\subsection{Propriétés~:}

Nous verrons ici plusieurs propriétés sur les automates de Glushkov, mais nous
n'en ferons pas la preuve, nous en donnerons une justification, mais pas une
réelle preuve (preuve disponible dans ce
document~\cite{DBLP:journals/tcs/CaronZ00}).

\vphantom{}

\begin{itemize}[label=\textbullet]
    \item La première propriété, plut\^{o}t évidente, est que les automates de
          Glushkov sont \textit{standards}, car par construction, il ne peut
          avoir qu'un seul état initial (0).

          \vphantom{}

    \item La deuxième est que l'automate a \(n + 1\) avec \(n\) le nombre de
      symboles de l'expression régulière. Le \(+ 1\) vient du fait que nous
      ajoutons un état \(0\) qui a des transitions vers les \(First\).

          \vphantom{}

    \item La troisième propriété un peu moins flagrante est que les automates de
      Glushkov sont accessibles et coaccessibles. C'est dû au fait que chaque
      symbole dans l'expression régulière est accessible et coaccessible et que
      cette propriété ne se perd pas lors de la transformation.

          \vphantom{}

    \item La quatrième propriété est que l'automate de Glushkov est homogène. Cela
      résulte de sa construction, car pour qu'un état aille sur un autre état,
      il faut qu'il ait dans ses \textit{Follow} \((a, n)\) avec \(a\) le
      symbole de la transition et \(n\) la valeur de l'état. Et étant donné que
      pour chaque couple \((b, m)\) il ne peut n'avoir que ce couple avec comme
      seconde valeur \(m\) alors la transition vers cet état sera toujours la
      même.

          \vphantom{}

    \item La cinquième propriété est que l'automate de Glushkov est un hamac. Car
      il est standard, accessible et coaccessible. Et que toutes ces orbites
      maximales sont fortement stables et transversales.

          \vphantom{}
\end{itemize}

\subsection{Conclusion}

L'algorithme de Glushkov est très puissant, puisqu'il permet de convertir une
expression régulière en automate, ce qui fait qu'on gagne les avantages des
deux structures. Avec les expressions régulières, on peut simplement décrire un
langage et avec les automates, on peut simplement savoir si un mot est reconnu.
Il est très utilisé en \textit{informatique}, parce que pour les humains, il
est plus simple de décrire un langage avec une expression régulière. Et les
machines comprennent très facilement les automates. Ce qui fait qu'il est
possible de faire des \textit{programmes informatiques} qui reconnaissent un
langage et exécutent des tâches à chaque mot.
