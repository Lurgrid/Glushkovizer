\section{Les automates}

Dans cette partie, nous parlerons des automates et plus particulièrement, nous
allons parler des automates sans \(\varepsilon\) transition (de célèbre
automate utilise des \(\varepsilon\)-transitions, comme ceux de
\textit{Thompson}, qui sont utilisés par nos ordinateurs). Pour autant les
automates que nous verrons sont tout aussi bons que ceux avec \(\varepsilon\)
transition.

\subsection{Définition}

Comme dit précédemment un automate est un objet mathématique reconnaissant un
langage, on notera \(M \in AFN(\Sigma, \eta)\) l'automate qui a pour transition
des valeurs dans \(\Sigma\), des valeurs \og{}d'état\fg{} dans \(\eta\). Et
\(AFN(\Sigma, \eta)\) l'ensemble des automates fini non déterministe de valeur
de transition dans \(\Sigma\) et de valeur d'état dans \(\eta\). On écrira donc
\(L(M)\) pour désigner le langage qu'il reconnait. Un automate est un tuple
qu'on peut écrire de cette forme \(M = (Q, I, F, \delta)\) avec~:

\begin{align}
    Q        & \subseteq \eta \quad \text{L'ensemble des états qui constitue l'automate} \\
    I        & \subseteq Q \quad \text{L'ensemble des états initiaux}                    \\
    F        & \subseteq Q \quad \text{L'ensemble des états finaux}                      \\
    \delta:~ & Q \times \Sigma \to 2^Q \quad \text{La fonction de transition}
\end{align}

Un automate peut se représenter à l'aide d'un graphe orienté, valué,
particulier, Par exemple si on veut représenter \(M = (\{q_1, q_2, q_3, q_4,
q_5\}, \{q_1\},\{q_2, q_3\}, \delta)\) avec \(M \in AFN(\Sigma, \eta)\),
\(\Sigma = \{0, 1\}\), \(\eta = \{q_1, q_2, q_3, q_4, q_5\}\) et \(\delta\)
défini comme ceci~:

\begin{align*}
    \delta(q_1, 0) & = \{q_2, q_4\} \\
    \delta(q_1, 1) & = \varnothing  \\
    \delta(q_2, 0) & = \varnothing  \\
    \delta(q_2, 1) & = \varnothing  \\
    \delta(q_3, 0) & = \{q_3\}      \\
    \delta(q_3, 1) & = \{q_4\}      \\
    \delta(q_4, 0) & = \{q_5\}      \\
    \delta(q_4, 1) & = \{q_3\}      \\
    \delta(q_5, 0) & = \{q_4\}      \\
    \delta(q_5, 0) & = \{q_5\}      \\
\end{align*}

\begin{figure}[H]
    \centering
    \captionsetup{type=figure,justification=centering}
    \begin{tikzpicture}
        \tikzset{
            ->,
            >=stealth',
            node distance=3cm,
            every state/.style={thick},
            initial text=$ $,
        }
        \node[state, initial] (q1) {$q_1$};
        \node[state, accepting, right of=q1] (q2) {$q_2$};
        \node[state, above of=q2] (q4) {$q_4$};
        \node[state, accepting, right of=q2] (q3) {$q_3$};
        \node[state, right of=q4] (q5) {$q_5$};

        \draw   (q1) edge[above] node{0} (q4)
        (q1) edge[below] node{0} (q2)
        (q4) edge[bend right, below] node{1} (q3)
        (q4) edge[above] node{0} (q5)
        (q3) edge[above] node{1} (q4)
        (q3) edge[loop right] node{0} (q3)
        (q5) edge[bend right, above] node{0} (q4)
        (q5) edge[loop right] node{1} (q5);
    \end{tikzpicture}
    \caption{
        Exemple de représentation graphique d'un automate.
    }\label{fig:automata}
\end{figure}

Dans la figure~\ref{fig:automata}, on peut voir que les états initiaux (ici que
\(q_1\)) ont une petite flèche qui pointe sur lui et que les états finaux ont
un double contour. Et que les transitions sont symbolisées par des flèches
entre les états et que ces flèches sont labellisées.

\vphantom{}

On parlera de l'inverse de l'automate \(M\) noté \(\overleftarrow{M}\) qui peut
être défini de cette façon~:

\begin{gather}
    \overleftarrow{M} = (Q, F, I, \delta') \quad \text{avec} \\
    M = (Q, I, F, \delta) \notag \\
    \forall (p, q) \in Q^2 ~|~ q \in \delta(p, a) \Rightarrow p \in \delta'(q, a) \notag
\end{gather}

Donc si on veut représenter l'inverse de l'automate représenté dans la
figure~\ref{fig:automata}, ça nous donnerait ceci~:

\begin{figure}[H]
    \centering
    \captionsetup{type=figure,justification=centering}
    \begin{tikzpicture}
        \tikzset{
            ->,
            >=stealth',
            node distance=3cm,
            every state/.style={thick},
            initial text=$ $,
        }
        \node[state, accepting] (q1) {$q_1$};
        \node[state, initial below, right of=q1] (q2) {$q_2$};
        \node[state, above of=q2] (q4) {$q_4$};
        \node[state, initial below, right of=q2] (q3) {$q_3$};
        \node[state, right of=q4] (q5) {$q_5$};

        \draw   (q4) edge[above] node{0} (q1)
        (q2) edge[below] node{0} (q1)
        (q3) edge[bend left, below] node{1} (q4)
        (q5) edge[above] node{0} (q4)
        (q4) edge[above] node{1} (q3)
        (q3) edge[loop right] node{0} (q3)
        (q4) edge[bend left, above] node{0} (q5)
        (q5) edge[loop right] node{1} (q5);
    \end{tikzpicture}
    \caption{
        Exemple de représentation graphique de l'inverse de l'automate de la
        figure~\ref{fig:automata}.
    }\label{fig:automata_invserse}
\end{figure}

On voit bien que l'apparence de l'automate ne change pas les transitions sont
juste inversées et les états initiaux sont devenus finaux et inversement.

\vphantom{}

On peut aussi étendre la fonction de transition \(\delta\) de manière qu'elle
ait comme signature~:

\[
    \delta_w: Q \times \Sigma^* \to 2^Q
\]

En la définissant récursivement de telle sorte~:

\begin{align}
    \delta_w(q, \varepsilon) & = \{q\}                                                                         \\
    \delta_w(q, a.w)         & = \bigcup_{q' \in \delta(q, a)} \delta_w(q', w) \quad \text{avec}~ a \in \Sigma
\end{align}

\begin{example}
    Voici donc quelques exemples de ce que ça nous donnerait si on reprend
    l'automate utilisé pour la représentation graphique
    (figure~\ref{fig:automata})~:

    \begin{gather}
        \delta_w(q_1, 00) = \varnothing \\
        \delta_w(q_1, 11) = \{q_3\} \\
        \delta_w(q_1, \varepsilon) = \{q_1\} \\
        \delta_w(q_1, 10 \cdot 1^*) = \{q_5\}
    \end{gather}
\end{example}

\vphantom{}

\subsubsection*{Propriétés~:}

Un automate est dit \textit{standard} à l'instant où il ne possède qu'un seul
état initial non ré-entrant, aussi défini comme ceci~:

\begin{gather}
    M = (Q, \{i\}, F, \delta) \quad \text{avec} \\
    \forall p \in Q, \forall a \in \Sigma ~|~ i \notin \delta(p, a) \notag \\
    M \in AFN(\Sigma, \eta) \notag
\end{gather}

Un automate est \textit{homogène} quand pour tous les états les transition
allant vers cet état ont même valeur. En d'autre terme quand il respecte cette
propriété~:

\begin{gather}
    M = (Q, I, F, \delta) \quad \text{avec} \\
    \forall (p, q, r) \in Q^3 ~|~ \exists (a, b) \in \Sigma^2 \land q \in \delta(p, a) \land q \in \delta(r, b) \Longrightarrow a = b \notag \\
    M \in AFN(\Sigma, \eta) \notag
\end{gather}

Un automate est qualifié d'\textit{accessible} lorsqu'en partant des initiaux,
on peut arriver sur tous les états qui le composent. C'est-à-dire qu'il valide
cette condition~:

\begin{gather}
    M = (Q, I, F, \delta) \quad \text{avec} \\
    \forall p \in Q, \exists w \in \Sigma^* ~|~ p \in \bigcup_{i \in I} \delta_w(i, w) \notag \\
    M \in AFN(\Sigma, \eta) \notag
\end{gather}

Un automate est considéré comme \textit{coaccessible} au moment où l'inverse de
cette automate est accessible. Ceci veut dire qu'il atteste de cette
particularité~:

\begin{gather}
    M = (Q, I, F, \delta) \quad \text{avec} \\
    accessible(\overleftarrow{M}) \notag \\
    M \in AFN(\Sigma, \eta) \notag
\end{gather}

Donc l'automate représenté sur la figure~\ref{fig:automata} est standard, non
homogène, accessible et coaccessible. Car il possède bien un unique état
initial (\(q_1\)), mais malheureusement \(q_3\), \(q_4\) et \(q_5\) ne respecte
pas la propriété pour être homogène, parce qu'ils ont des transitions allant
vers eux avec des valeurs différentes. De plus tous ses états sont par
accessible depuis l'état initial et son inverse est, lui-même aussi,
accessible.

\vphantom{}

Les automates pouvant être représentés à l'aide de graphe, on peut étendre les
propriétés sur les graphes aux automates. Par exemple, on pourra parler des
composantes fortement connexes d'un automate. Autrement dit en partant de
n'importe quel état, on peut arriver à tous les autres états. Ainsi ça veut
dire qu'un automate fortement connexe vérifierait ceci~:

\begin{gather}
    M = (Q, I, F, \delta) \quad \text{avec} \\
    \forall (p, q) \in Q^2, \exists w \in \Sigma^* ~|~ q \in \delta_w(p, w) \notag \\
    M \in AFN(\Sigma, \eta) \notag
\end{gather}

\subsection{Fonction sur les automates}

Une des fonctions la plus importante sur les automates est \(accept\) qui test
si le mot est reconnu par l'automate. C'est-à-dire que si on prend le chemin
décrit par le mot donner en argument, on arrive sur un ou plusieurs états
finaux. Elle a alors pour signature~:

\[
    accept: AFN(\Sigma, \eta) \times \Sigma^* \to \mathbb{B}
\]

Elle peut être définie simplement comme ceci~:

\begin{align}
    accept(M, w) = \exists q \in \bigcup_{p \in I} \delta_w(p, w) ~|~ q \in F
\end{align}

\subsection{Conclusion}

Comme nous venons de voir les automates sont des outils puissants pour
reconnaitre des mots d'un langage. L'un de leur plus grande force est leur
simplicité, toutes les opérations sur les automates peuvent donc être
automatisé. Ce qui fait que cet objet est très intéressant dans le monde de
l'\textit{informatique}. En revanche l'un de ses points faibles est sa
représentation, il est difficile de représenter des très gros automates
contrairement aux expressions régulières. Il serait alors intéressant de
pouvoir convertir une expression régulière en automate. C'est ce que nous
allons voir dans la prochaine section.