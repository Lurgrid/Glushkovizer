\section{Les automates}

Dans cette partie, nous parlerons des automates et plus particulièrement, nous
allons parler des automates sans \(\varepsilon\)-transition (de célèbres
automates utilisent des \(\varepsilon\)-transitions, comme ceux de
\textit{Thompson}, qui sont utilisés par nos ordinateurs). Pour autant, les
automates que nous verrons sont tout aussi bons que ceux avec
\(\varepsilon\)-transition.

\subsection{Définition}

Comme dit précédemment, un automate est un objet mathématique reconnaissant un
langage. On notera \(M \in AFN(\Sigma, \eta)\) l'automate qui a pour transition
des valeurs dans \(\Sigma\), des valeurs \og{}d'état\fg{} dans \(\eta\). Et
\(AFN(\Sigma, \eta)\) l'ensemble des automates finis non déterministes de
valeur de transition dans \(\Sigma\) et de valeur d'état dans \(\eta\). On
écrira donc \(L(M)\) pour désigner le langage qu'il reconnait. Un automate est
un tuple qu'on peut écrire de cette forme \(M = (Q, I, F, \delta)\) avec~:

\begin{align*}
    Q & \subseteq \eta \quad \text{L'ensemble des états qui constituent
    l'automate}                                                                \\
    I & \subseteq Q \quad \text{L'ensemble des états initiaux}          \\
    F & \subseteq Q \quad \text{L'ensemble des états finaux}            \\
    \delta:~ & Q \times \Sigma \to 2^Q \quad \text{La fonction de transition}
\end{align*}

Un automate peut se représenter à l'aide d'un graphe orienté, valué,
particulier. Par exemple si on veut représenter \(M = (\{q_1, q_2, q_3, q_4,
q_5\}, \{q_1\},\{q_2, q_3\}, \delta)\) avec \(M \in AFN(\Sigma, \eta)\),
\(\Sigma = \{0, 1\}\), \(\eta = \{q_1, q_2, q_3, q_4, q_5\}\) et \(\delta\)
défini comme ceci~:

\begin{align*}
    \delta(q_1, 0) & = \{q_2, q_4\} & \delta(q_3, 1) & = \{q_4\} \\
    \delta(q_1, 1) & = \varnothing  & \delta(q_4, 0) & = \{q_5\} \\
    \delta(q_2, 0) & = \varnothing  & \delta(q_4, 1) & = \{q_3\} \\
    \delta(q_2, 1) & = \varnothing  & \delta(q_5, 0) & = \{q_4\} \\
    \delta(q_3, 0) & = \{q_3\}      & \delta(q_5, 1) & = \{q_5\} \\
\end{align*}

\begin{figure}[H]
    \centering
    \captionsetup{type=figure,justification=centering}
    \begin{tikzpicture}
        \tikzset{
            ->,
            >=stealth',
            node distance=3cm,
            every state/.style={thick},
            initial text=$ $,
        }
        \node[state, initial] (q1) {$q_1$};
        \node[state, accepting, right of=q1] (q2) {$q_2$};
        \node[state, above of=q2] (q4) {$q_4$};
        \node[state, accepting, right of=q2] (q3) {$q_3$};
        \node[state, right of=q4] (q5) {$q_5$};

        \draw   (q1) edge[above] node{0} (q4)
        (q1) edge[below] node{0} (q2)
        (q4) edge[bend right, below] node{1} (q3)
        (q4) edge[above] node{0} (q5)
        (q3) edge[above] node{1} (q4)
        (q3) edge[loop right] node{0} (q3)
        (q5) edge[bend right, above] node{0} (q4)
        (q5) edge[loop right] node{1} (q5);
    \end{tikzpicture}
    \caption{
        Exemple de représentation graphique d'un automate.
    }\label{fig:automata}
\end{figure}

Dans la Figure~\ref{fig:automata}, on peut voir que les états initiaux (ici que
\(q_1\)) ont une petite flèche qui pointe sur eux et que les états finaux ont
un double contour. Et que les transitions sont symbolisées par des flèches
entre les états et que ces flèches sont labellisées.

% \vphantom{}

% On parlera de l'inverse de l'automate \(M\) noté \(\overleftarrow{M}\) qui peut
% être défini de cette façon~:

% \begin{gather*}
%     \overleftarrow{M} = (Q, F, I, \delta') \quad \text{avec} \\
%     M = (Q, I, F, \delta) \notag \\
%     \forall (p, q) \in Q^2 ~|~ q \in \delta(p, a) \Rightarrow p \in \delta'(q, a) \notag
% \end{gather*}

% Donc si on veut représenter l'inverse de l'automate représenté dans la
% Figure~\ref{fig:automata}, ça nous donnerait ceci~:

% \begin{figure}[H]
%     \centering
%     \captionsetup{type=figure,justification=centering}
%     \begin{tikzpicture}
%         \tikzset{
%             ->,
%             >=stealth',
%             node distance=3cm,
%             every state/.style={thick},
%             initial text=$ $,
%         }
%         \node[state, accepting] (q1) {$q_1$};
%         \node[state, initial below, right of=q1] (q2) {$q_2$};
%         \node[state, above of=q2] (q4) {$q_4$};
%         \node[state, initial below, right of=q2] (q3) {$q_3$};
%         \node[state, right of=q4] (q5) {$q_5$};

%         \draw   (q4) edge[above] node{0} (q1)
%         (q2) edge[below] node{0} (q1)
%         (q3) edge[bend left, below] node{1} (q4)
%         (q5) edge[above] node{0} (q4)
%         (q4) edge[above] node{1} (q3)
%         (q3) edge[loop right] node{0} (q3)
%         (q4) edge[bend left, above] node{0} (q5)
%         (q5) edge[loop right] node{1} (q5);
%     \end{tikzpicture}
%     \caption{
%         Exemple de représentation graphique de l'inverse de l'automate de la
%         Figure~\ref{fig:automata}.
%     }\label{fig:automata_invserse}
% \end{figure}

% On voit bien que l'apparence de l'automate ne change pas les transitions sont
% juste inversées et les états initiaux sont devenus finaux et inversement.

\vphantom{}

On peut aussi étendre la fonction de transition \(\delta\) de manière qu'elle
ait comme signature~:

\[
    \delta: Q \times \Sigma^* \to 2^Q
\]

En la définissant récursivement de telle sorte~:

\begin{align*}
    \delta(q, \varepsilon) & = \{q\}                                                                       \\
    \delta(q, a \cdot w)   & = \bigcup_{q' \in \delta(q, a)} \delta(q', w) \quad \text{avec}~ a \in \Sigma
\end{align*}

\begin{example}
    Voici donc quelques exemples de ce que ça nous donnerait si on reprend
    l'automate utilisé pour la représentation graphique
    (Figure~\ref{fig:automata})~:

    \begin{gather*}
        \delta(q_1, 00) = \varnothing \\
        \delta(q_1, 11) = \{q_3\} \\
        \delta(q_1, \varepsilon) = \{q_1\} \\
        \delta(q_1, 10 \cdot 1^*) = \{q_5\}
    \end{gather*}
\end{example}

\vphantom{}

\begin{definition}
    Un automate est dit \textit{standard} à l'instant où il ne possède qu'un
    seul état initial non ré-entrant, aussi défini comme ceci~:

    \begin{gather*}
        M = (Q, \{i\}, F, \delta) \quad \text{avec} \\
        \forall p \in Q, \forall a \in \Sigma ~|~ i \notin \delta(p, a) \notag \\
        M \in AFN(\Sigma, \eta) \notag
    \end{gather*}
\end{definition}

\begin{definition}
    Un automate est \textit{homogène} quand, pour tous les états, les
    transitions allant vers cet état ont la même valeur. En d'autres termes,
    quand il respecte cette propriété~:

    \begin{gather*}
        M = (Q, I, F, \delta) \quad \text{avec} \\
        \forall (p, q, r) \in Q^3, \exists (a, b) \in \Sigma^2 ~|~ q \in \delta(p, a) \land q \in \delta(r, b) \Longrightarrow a = b \notag \\
        M \in AFN(\Sigma, \eta) \notag
    \end{gather*}
\end{definition}

\begin{definition}
    Un automate est qualifié d'\textit{accessible} lorsqu'en partant des
    initiaux, on peut arriver sur tous les états qui le composent. C'est-à-dire
    qu'il valide cette condition~:

    \begin{gather*}
        M = (Q, I, F, \delta) \quad \text{avec} \\
        \forall p \in Q, \exists w \in \Sigma^* ~|~ p \in \bigcup_{i \in I} \delta(i, w) \notag \\
        M \in AFN(\Sigma, \eta) \notag
    \end{gather*}
\end{definition}

\begin{definition}
    Un automate est considéré comme \textit{coaccessible} au moment où, de tous
    les états, on peut arriver à un état final. Ceci veut dire qu'il atteste de
    cette particularité~:

    \begin{gather*}
        M = (Q, I, F, \delta) \quad \text{avec} \\
        \forall p \in Q, \exists w \in \Sigma^* ~|~ F \cap \delta(p, w) \neq \varnothing \notag \\
        M \in AFN(\Sigma, \eta) \notag
    \end{gather*}
\end{definition}

\begin{definition}
    Un automate est dit \textit{déterministe} quand tous ses états vont au
    maximum à un état par symbole. Autrement dit qu'il valide cette propriété~:

    \begin{gather*}
        M = (Q, I, F, \delta) \quad \text{avec} \\
        |I| = 1 \land \forall q \in Q, \forall a \in \Sigma, | \delta(q, a) | \leq 1\\
        M \in AFN(\Sigma, \eta) \notag
    \end{gather*}

    On parlera de \textit{déterministe complet} lorsque tous ses états vont sur
    un état par symbole. C'est-à-dire qu'il respecte cette condition~:

    \begin{gather*}
        M = (Q, I, F, \delta) \quad \text{avec} \\
        |I| = 1 \land \forall q \in Q, \forall a \in \Sigma, | \delta(q, a) | = 1\\
        M \in AFN(\Sigma, \eta) \notag
    \end{gather*}
\end{definition}

\begin{example}
    Donc, l'automate représenté sur la Figure~\ref{fig:automata} est standard,
    non homogène, accessible et coaccessible. Car il possède bien un unique
    état initial (\(q_1\)), mais \(q_3\), \(q_4\) et \(q_5\) ne respecte pas la
    propriété pour être homogène, parce qu'ils ont des transitions allant vers
    eux avec des valeurs différentes. De plus, tous ses états sont accessibles
    depuis l'état initial. Et son inverse est, lui-même aussi, accessible et il
    n'est pas déterministe.
\end{example}

\begin{definition}
    Nous parlerons de sous-automate pour parler d'une \og{}région\fg{} d'un
    automate. \(N\) est un sous automate de \(M\) qu'on notera \(N \subseteq
    M\), s'il vérifie cette propriété~:

    \begin{gather*}
        N = (Q', I', F', \delta') \quad \text{avec} \\
        Q' \subseteq Q \land I' \subseteq I \land F' \subseteq F \\
        \delta' \text{ est une restriction de } \delta \\
        M = (Q, I, F, \delta) \land (M, N) \in (AFN(\Sigma, \eta))^2 \notag
    \end{gather*}
\end{definition}

Les automates pouvant être représentés à l'aide de graphes, on peut étendre les
propriétés sur les graphes aux automates. Par exemple, on pourra parler des
composantes fortement connexes d'un automate. Autrement dit, en partant de
n'importe quel état, on peut arriver à tous les autres états. Ainsi, ça veut
dire qu'un automate fortement connexe vérifierait ceci~:

\begin{gather*}
    M = (Q, I, F, \delta) \quad \text{avec} \\
    \forall (p, q) \in Q^2, \exists w \in \Sigma^* ~|~ q \in \delta(p, w) \notag \\
    M \in AFN(\Sigma, \eta) \notag
\end{gather*}

\begin{definition}
    Une autre notion qui est présente sur les graphes que nous allons adapter
    sur les automates est la notion de \textit{hamac}. Nous dirons qu'un
    automate est un \textit{hamac} à l'instant où il est standard, accessible
    et coaccessible (nous gardons le nom \textit{hamac} pour une raison de
    compréhension). Ceci veut dire qu'il peut être décrit comme ceci~:

    \begin{gather*}
        M = (Q, I, F, \delta) \quad \text{avec} \\
        standard(M) \land accessible(M) \land coaccessible(M) \\
        M \in AFN(\Sigma, \eta) \notag
    \end{gather*}
\end{definition}

\begin{conjecture}
    La représentation d'un automate \textit{hamac} en graphe. C'est-à-dire en
    enlevant le concept d'états initiaux, finaux et en ajoutant un nouvel état
    où tous les finaux ont une transition vers lui. Et en retirant les valeurs
    de transition. Le graphe résultant est lui-même un \textit{hamac} avec
    comme \textit{racine} l'état initial et l'\textit{anti-racine} l'état
    ajouté lors de la transformation.
\end{conjecture}

\begin{definition}
    Une autre idée empruntée au graphe est la notion d'\textit{orbite}. Nous
    dirons qu'un sous-automate est une \textit{orbite}, si pour tout couple
    d'état \(i\) et \(t\), il existe un mot non vide permettant d'aller de
    \(i\) à \(t\). Aussi défini comme ceci~:

    \begin{gather*}
        \mathcal{O} = (Q, I, F, \delta) \quad \text{avec} \\
        \forall (p, q) \in Q^2, \exists w \in \Sigma^* \setminus \{\varepsilon\} ~|~ q \in \Sigma(p, w) \\
        \mathcal{O} \subseteq M \land M \in AFN(\Sigma, \eta)
    \end{gather*}

    \noindent Dans la même idée, nous parlerons d'\textit{orbite maximale}
    lorsque l'orbite n'est incluse dans aucune autre orbite. En d'autres
    termes, que l'orbite est fortement connexe sans prendre les chemins
    triviaux
    (mot vide).
\end{definition}

\begin{definition}
    Nous noterons \(In(\mathcal{O})\) et \(Out(\mathcal{O})\) respectivement
    l'ensemble des portes d'entrée et l'ensemble des portes de sortie de
    l'orbite \(\mathcal{O}\). Qui peuvent être définies de cette façon~:

    \begin{gather*}
        In(\mathcal{O}) = \{p \in Q' ~|~ \exists a \in \Sigma, \exists q \in Q,
        p \in \delta(q, a)\} \\
        Out(\mathcal{O}) = \{p \in Q' ~|~ \exists a \in \Sigma, \exists q \in
        Q, q \in \delta(p, a)\} \\
        \text{avec} \\
        \mathcal{O} = (Q', I', F', \delta') \land M = (Q, I, F, \delta) \\
        \mathcal{O} \subseteq M \land M \in AFN(\Sigma, \eta)
    \end{gather*}
\end{definition}

\begin{definition}
    Avec ceci, on peut définir ce qu'est une \textit{orbite stable}. Une orbite
    est dite \textit{stable} quand pour toutes les sorties, il existe une
    transition vers toutes les entrées. C'est-à-dire que l'orbite vérifie
    ceci~:

    \begin{gather*}
        \forall q \in Out(\mathcal{O}), \exists a \in \Sigma ~|~ \delta(q, a) \cap In(\mathcal{O}) \neq \varnothing \\
        \text{avec} \\
        \mathcal{O} = (Q, I, F, \delta) \subseteq M \land M \in AFN(\Sigma, \eta)
    \end{gather*}

    \noindent On la qualifiera même de \textit{fortement stable} lorsqu'en
    supprimant toutes les transitions de portes des sorties vers les portes
    d'entrée, les orbites maximales de l'orbite sont \textit{fortement stables}.
\end{definition}

\begin{definition}
    De même, on dira qu'une orbite est \textit{transversale} si toutes les
    entrées viennent des mêmes états et que toutes les sorties vont aux mêmes
    états. Autrement dit, que l'orbite valide cette propriété~:

    \begin{gather*}
        \forall (p, q) \in (Out(\mathcal{O}))^2, \bigcup_{a \in \Sigma} \delta(p, a) = \bigcup_{a \in \Sigma} \delta(q, a) \\
        \forall (p, q) \in (In(\mathcal{O}))^2, \{r \in Q ~|~ \exists a \in \Sigma, p \in \delta(r, a)\} = \{r \in Q ~|~ \exists a \in \Sigma, q \in \delta(r, a)\} \\
        \text{avec} \\
        \mathcal{O} \subseteq M = (Q, I, F, \delta) \land M \in AFN(\Sigma, \eta)
    \end{gather*}

    \noindent Elle sera même \textit{fortement transversale} lorsqu'en
    supprimant toutes les transitions de portes des sorties vers les portes
    d'entrée, les orbites maximales de l'orbite sont
    \textit{fortement transversales}.
\end{definition}

\subsection{Fonction sur les automates}

Une des fonctions sur les automates est \(accept\) qui test si le mot est
reconnu par l'automate. C'est-à-dire que si on prend le chemin décrit par le
mot donner en argument, on arrive sur un ou plusieurs états finaux. Elle a
alors pour signature~:

\[
    accept: AFN(\Sigma, \eta) \times \Sigma^* \to \mathbb{B}
\]

Elle peut être définie simplement comme ceci~:

\begin{align*}
    accept(M, w) = \exists q \in \bigcup_{p \in I} \delta(p, w) ~|~ q \in F
\end{align*}

\subsection{Conclusion}

Comme nous venons de voir, les automates sont des outils puissants pour
reconnaitre des mots d'un langage. L'une de leurs grandes forces est leur
simplicité. Toutes les opérations sur les automates peuvent donc être
automatisées. Ce qui fait que cet objet est très intéressant dans le monde de
l'informatique. En revanche, l'un de ses points faibles est sa représentation.
Il est difficile de représenter des très gros automates, contrairement aux
expressions régulières. Il serait alors intéressant de pouvoir convertir une
expression régulière en automate. C'est ce que nous allons voir dans la
prochaine section.
