\section{Prélude}

Pour la compréhension de l'ensemble de ce document, nous avons besoin de
plusieurs notions de théorie des langages. C'est donc pourquoi cette partie,
nous allons étudier les différentes notions nécessaires. Dans un premier temps,
nous allons définir ce qu'est un mot et quelles sont les opérations sur les
mots. Enfin dans un second temps, nous définirons ce qu'est un langage et
quelles opérations sont munis sur les langages.

\subsection{Les mots}

Un \textit{alphabet} \(\Sigma\) est un ensemble fini de symbole non vide. Un
\textit{mot} est une suite finie de symbole sur un alphabet \(\Sigma\), le mot
composé de zéro symbole est appelé `mot vide' est noté \(\varepsilon\).

\begin{example}
    \begin{gather}
        \Sigma = \{a, b, c, d\} \\
        w = abbcdda
    \end{gather}
\end{example}

On parlera de la \textit{longueur d'un mot} \(w\) noté \(|w|\) pour désigner le
nombre de symboles qui le compose. De même, on notera \({|w|}_a\) pour parler
du nombre de \(a\) dans le mot \(w\).

\begin{example}
    \begin{gather}
        w = abbcdda \notag \\
        |w| = 7 \\
        {|w|}_d = 2
    \end{gather}
\end{example}

Une des opérations les plus essentiels sur les mots est la concaténation de
mots. On notera donc la concaténation de deux mots \(u = a_1 \cdots a_n\) et
\(v = b_1 \cdots b_n\) par \(u \cdot v\). Qui est ainsi égal à \(u \cdot v =
a_1 \cdots a_n b_1 \cdots b_n\). On définit l'ensemble des mots sur \(\Sigma\)
par \(\Sigma ^ *\). On notera que~:

\begin{itemize}[label=\textbullet]
    \item La concaténation est associative \((w \cdot u) \cdot v = w \cdot (u \cdot v)\).
    \item La concaténation admet un élément neutre \(u \cdot \varepsilon = \epsilon \cdot
          u = u\).
\end{itemize}

\noindent Ce qui implique que l'ensemble \(\Sigma ^ *\) muni de la concaténation
\((\Sigma, \cdot)\) forme un monoïde.

\begin{example}
    \begin{gather}
        u = abab \\
        v = cdcd \\
        u \cdot v = ababcdcd
    \end{gather}
\end{example}

Grâce à cette opération sur les mots, on peut définir ce qu'est un
\textit{facteur}. Un facteur \(u\) d’un mot \(w\) est une suite extraite de la
suite de lettre qui composent le mot \(w\). Autrement dit \(u\) est un sous mot
de \(w\) si \(\exists (v, x) \in (\Sigma ^ *)^2 ~|~ w = v \cdot u \cdot x\), de
plus~:

\begin{itemize}[label=\textbullet]
    \item On parlera de \textit{préfixe} quand \(v = \varepsilon\).
    \item On parlera de \textit{suffixe} quand \(x = \varepsilon\).
    \item Enfin, on parlera de \textit{facteur propre} quand \(v \neq \varepsilon \land x
          \neq \varepsilon\).
\end{itemize}

\noindent On remarquera que \(\varepsilon\) est \textit{préfixe}, \textit{suffixe} et
\textit{facteur} de tout mot.