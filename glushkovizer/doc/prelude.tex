\section{Prélude}

Pour la compréhension de l'ensemble de ce document, nous avons besoin de
plusieurs notions de théorie des langages. C'est donc pourquoi cette partie,
nous allons étudier les différentes notions nécessaires. Dans un premier temps,
nous allons définir ce qu'est un mot et quelles sont les opérations sur les
mots. Enfin dans un second temps, nous définirons ce qu'est un langage et
quelles opérations sont munis sur les langages.

\subsection{Les mots}

\begin{definition}
    Un \textit{alphabet} \(\Sigma\) est un ensemble fini de symbole non vide. Un
    \textit{mot} est une suite finie de symbole sur un alphabet \(\Sigma\), le mot
    composé de zéro symbole est appelé \textit{mot vide} est noté \(\varepsilon\).
\end{definition}

\begin{example}
    \begin{gather*}
        \Sigma = \{a, b, c, d\} \\
        w = abbcdda
    \end{gather*}
\end{example}

\begin{definition}
    On parlera de la \textit{longueur d'un mot} \(w\) noté \(|w|\) pour désigner le
    nombre de symboles qui le compose.
    % De même, on notera \({|w|}_a\) pour parler du nombre de \(a\) dans le mot
    % \(w\).
\end{definition}

\begin{example}
    \begin{gather*}
        w = abbcdda \notag \\
        |w| = 7 \\
        % {|w|}_d = 2
    \end{gather*}
\end{example}

\begin{definition}
    Une des opérations les plus essentiels sur les mots est la
    \textit{concaténation} de mots. On notera donc la concaténation de deux mots
    \(u = a_1 \cdots a_n\) et \(v = b_1 \cdots b_n\) par \(u \cdot v\). Qui est
    ainsi égal à \(u \cdot v = a_1 \cdots a_n b_1 \cdots b_n\). On définit
    l'ensemble des mots sur \(\Sigma\) par \(\Sigma ^ *\). On notera que~:

    \begin{itemize}[label=\textbullet]
        \item La concaténation est associative \((w \cdot u) \cdot v = w \cdot (u \cdot v)\).
        \item La concaténation admet un élément neutre \(u \cdot \varepsilon = \epsilon \cdot
              u = u\).
    \end{itemize}

    \noindent Ce qui implique que l'ensemble \(\Sigma ^ *\) muni de la concaténation
    \((\Sigma, \cdot)\) forme un monoïde.
\end{definition}

\begin{example}
    \begin{gather*}
        u = abab \\
        v = cdcd \\
        u \cdot v = ababcdcd
    \end{gather*}
\end{example}

% \begin{definition}
%     Grâce à cette opération sur les mots, on peut définir ce qu'est un
%     \textit{facteur}. Un facteur \(u\) d'un mot \(w\) est une suite extraite de la
%     suite de lettre qui composent le mot \(w\). Autrement dit \(u\) est un sous mot
%     de \(w\) si \(\exists (v, x) \in (\Sigma ^ *)^2 ~|~ w = v \cdot u \cdot x\), de
%     plus~:

%     \begin{itemize}[label=\textbullet]
%         \item On parlera de \textit{préfixe} quand \(v = \varepsilon\).
%         \item On parlera de \textit{suffixe} quand \(x = \varepsilon\).
%         \item Enfin, on parlera de \textit{facteur propre} quand \(v \neq \varepsilon \land x
%               \neq \varepsilon\).
%     \end{itemize}

%     \noindent On remarquera que \(\varepsilon\) est~: \textit{préfixe}, \textit{suffixe} et
%     \textit{facteur} de tout mot.
% \end{definition}

% \begin{example}
%     \begin{gather*}
%         w = abbcdda \\
%         u = bcd \\
%         y = abb \\
%         w = ab \cdot u \cdot da\\
%         u \text{ est donc un facteur propre} \notag \\
%         w = y \cdot cdda \quad \\
%         y \text{ est un facteur et un préfixe} \notag
%     \end{gather*}
% \end{example}

% \begin{definition}
%     De plus aussi grâce à la concaténation, nous pouvons définir le \textit{miroir}
%     d'un mot \(w\) noté \(\overleftarrow{w}\). Qui est ainsi défini récursivement
%     comme ceci~:

%     \begin{gather*}
%         \overleftarrow{\varepsilon} = \varepsilon \\
%         \overleftarrow{a} = a \\
%         \overleftarrow{u \cdot a} = a \cdot \overleftarrow{u} \\
%         \text{Avec } a \in \Sigma \text{ et } u \in \Sigma ^ * \notag
%     \end{gather*}

%     \noindent On remarquera qu'un mot est un palindrome si \(u = \overleftarrow{u}\).
% \end{definition}

% \begin{example}
%     \begin{gather*}
%         w = abbcdda \\
%         \overleftarrow{w} = addcbba
%     \end{gather*}
% \end{example}

\subsection{Les langages}

\begin{definition}
    Un \textit{langage} \(L\) est un ensemble de mot sur un alphabet fini
    \(\Sigma\). On appellera \textit{langage vide} le langage ne comportant aucun
    mot, ainsi défini comme ceci~: \(L = \varnothing\).
\end{definition}

\begin{example}
    \begin{gather*}
        \Sigma = \{a, b, c, d\} \\
        L_1 = \{a, aa, bc, da, \varepsilon\} \\
        L_2 = \varnothing
    \end{gather*}
\end{example}

% \begin{definition}
%     L'équivalant de la longueur d'un mot sur les langages est donc le
%     \textit{cardinal} du langage. On remarquera qu'un langage n'est pas forcément
%     fini et qu'alors son \textit{cardinal} peut être infini.
% \end{definition}

% \begin{example}
%     \begin{gather*}
%         \Sigma = \{a, b, c, d\} \\
%         L_1 = \{a, aa, bc, da, \varepsilon\} \\
%         L_2 = \varnothing \\
%         | L_1 | = 5 \\
%         | L_2 | = 0
%     \end{gather*}
% \end{example}

\begin{definition}
    L'une des opérations sur les langages est l'\textit{union}. On parlera de
    l'union de langage noté \(L_1 \cup L_2\) et défini comme ceci~:

    \begin{gather*}
        L_1 \cup L_2 = \{w \in \Sigma ^ * ~|~ w \in L_1 \lor w \in L_2\}
    \end{gather*}

    \noindent On notera que l'union est associative, commutative et admet un
    élément neutre (\(\varnothing\)).
\end{definition}

% \begin{definition}
%     À l'image de l'union de langages, nous avons aussi l'\textit{intersection} de
%     langage noté \(L_1 \cap L_2\) et donc défini comme cela~:

%     \begin{gather*}
%         L_1 \cap L_2 = \{w \in \Sigma ^ * ~|~ w \in L_1 \land w \in L_2\}
%     \end{gather*}

%     \noindent On notera qu'elle aussi est associative, commutative et admet
%     aussi un élément neutre (\(\Sigma ^ *\)). On remarquera que \(\varnothing\) est
%     un élément absorbant.
% \end{definition}

\begin{example}
    \begin{gather*}
        \Sigma = \{a, b, c, d\} \\
        L_1 = \{\varepsilon, a, aa, bc, da\} \\
        L_2 = \{d, aa, cd\} \\
        L_1 \cup L_2 = \{\varepsilon, a, d, aa, cd, bc, da\} \\
        % L_1 \cap L_2 = \{aa\}
    \end{gather*}
\end{example}

\begin{definition}
    Un autre opération sur les langages est la \textit{concaténation}. Elle est
    bien sûr définie en utilisant la concaténation des mots qui compose les
    langages. Cette opération est ainsi définie comme ceci~:

    \begin{gather*}
        L_1 \cdot L_2 = \{u \cdot v ~|~ u \in L_1, v \in L_2\}
    \end{gather*}

    \noindent On remarquera qu'elle est associative, pas commutative et admet un
    élément neutre (\(\{\varepsilon\}\)). Et que \(\varnothing\) est aussi un
    élément absorbant pour cette opération.
\end{definition}

\begin{example}
    \begin{gather*}
        \Sigma = \{a, b, c, d\} \\
        L_1 = \{\varepsilon, a, aa\} \\
        L_2 = \{d, cc\} \\
        L_1 \cdot L_2 = \{d, ad, aad, cc, acc, aacc\}
    \end{gather*}
\end{example}

\begin{definition}
    Par extension, on définit la \textit{copie n-ième} d'un langage \(L\) notée
    \(L^n\) et défini récursivement comme ceci~:

    \begin{gather*}
        L^0 = \{\varepsilon\} \\
        L^n = L^{n - 1} \cdot L
    \end{gather*}

    \noindent On remarquera que \(\varnothing^0 = \{\varepsilon\}\).
\end{definition}

\begin{example}
    \begin{gather*}
        \Sigma = \{a, b\} \\
        L = \{\varepsilon, a\} \\
        L^3 = \{\varepsilon, a, aa, aaa\}
    \end{gather*}
\end{example}

\begin{definition}
    Grâce à cette opération, on peut définir l'\textit{étoile} d'un langage notée
    \(L^*\). Qui peut être défini comme ceci~:

    \begin{gather*}
        L^* = \bigcup_{i \geq 0} L^i
    \end{gather*}
\end{definition}

\begin{example}
    \begin{gather*}
        L = \{a\} \\
        L^* = \{\varepsilon\} \cup \{a\} \cup \{aa\} \cup \cdots
    \end{gather*}
\end{example}

\subsection{Conclusion}

Nous avons défini les concepts de \textit{mot}, de \textit{langage} et
l'ensemble des opérations applicables à ces objets. Bien que nous n'ayons
couvert qu'une partie des opérations possibles, nous vous encourageons à
consulter un cours de théorie des langages pour obtenir des informations plus
détaillées. Nous vous conseillons ces
ressources~:~\cite{Harrison1978},~\cite{Autebert1994} et~\cite{Hopcroft2007}.