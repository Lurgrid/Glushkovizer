\section{Les expressions régulières}

Dans cette section, nous parlerons d'expressions régulières (\textit{ER}). Nous
allons nous concentrer sur un type bien particulier d'expressions régulières
qui ne seront pas les expressions régulières que nous pouvons voir plus
quotidiennement dans le domaine de l'informatique, les expressions régulières
\textit{UNIX}. Mais plutôt une version plus simple de celles-ci.

\subsection{Définition}

Nous allons noter une expression régulière \(E(\Sigma)\), c'est-à-dire une
expression régulière où les symboles sont inclus dans l'ensemble \(\Sigma\).
Cette expression reconnait un langage qu'on pourra appeler \(L(E(\Sigma))\).
Nous pouvons définir une expression régulière récursivement de cette manière~:

\begin{align}
    E                      & = \varepsilon                          \\
    E                      & = a                                    \\
    E                      & = F + G                                \\
    E                      & = F \cdot G                            \\
    E                      & = F^*                                  \\
    E                      & = (F)\label{align:parenthese}          \\
    \text{avec } (E, F, G) & \in (E(\Sigma))^3,~ a \in \Sigma\notag
\end{align}

On notera que \(*\) est prioritaire sur \(\cdot\) qui est lui-même prioritaire
sur \(+\) et qu'ils sont tous deux associatifs à gauche. On comprend donc
pourquoi l'équation~(\ref{align:parenthese}) existe, elle est là pour des
raisons de priorité. Il est alors évident de calculer les diverses fonctions
sur celle-ci, c'est pour cela qu'on ne précisera pas son
calcul\label{text:parenthese}. On peut définir chaque équation comme ceci~:

\vphantom{}

\begin{itemize}
    \item[\textbullet] \textbf{\(E = \varepsilon\) (epsilon)~:}
        Représente le mot vide, de ce fait un mot de longueur zéro. Il en vient
        que~:

        \begin{align}
            L(E) & = \{\varepsilon\}
        \end{align}

        Il peut être parfois représenté par `\$'.

        \vphantom{}

    \item[\textbullet] \textbf{\(E = a\)~:} Représente un symbole
        présent dans l'ensemble \(\Sigma\). Il en vient que~:

        \begin{align}
            L(E) & = \{a\}
        \end{align}

        \vphantom{}

    \item[\textbullet] \textbf{\(E = F + G\)~:} Représente l'union
        des deux expressions régulières \(F\) et \(G\). Il en
        vient que~:

        \begin{align}
            L(E) & = L(F) \cup L(G)
        \end{align}

        Par abus de langage, on peut aussi dire \(F\) `ou' \(G\) pour représenter cette
        union.

        \vphantom{}

    \item[\textbullet] \textbf{\(E = F \cdot G\)~:}
        Représente la concaténation deux des deux expressions régulières
        \(F\) et \(G\). Il en vient que~:

        \begin{align}
            L(E) & = L(F) \cdot L(G)
        \end{align}

        \vphantom{}

    \item[\textbullet] \textbf{\(E = F^* \)~:} Représente la
        répétition infinie de \(F(\Sigma)\), cette répétition incluant,
        puissance \(0\) et donc le mot vide. Il en vient que~:

        \begin{align}
            L(E) & = (L(F))^*
        \end{align}
\end{itemize}

\begin{example}
    On comprendra ainsi que l'expression \(E(\mathbb{A}) = a+c \cdot d\) avec
    \(\mathbb{A} = \{a, b, c, d\}\), dénote le langage \(L(E(\mathbb{A})) = \{a,
    cd\}\). Car on peut représenter \(E(\mathbb{A})\) comme ceci~:

    \begin{figure}[H]
        \centering
        \captionsetup{type=figure,justification=centering}
        \begin{tikzpicture}[
                mycircle/.style={
                        draw,
                        circle,
                        minimum height=.75cm,
                        minimum width=.75cm
                    },
                mysquare/.style={
                        draw,
                        rectangle,
                        minimum height=.5cm,
                        minimum width=.5cm,
                    }
            ]
            \node[mysquare] (plus) {+}
            child {node[mycircle] (a) {a}}
            child {
                    node[mysquare] (point) {.}
                    child {node[mycircle] (c) {c}}
                    child {node[mycircle] (d) {d}}
                };

            \node[right=3mm of d] {\(\{d\}\)};
            \node[left=3mm of c] {\(\{c\}\)};
            \node[right=3mm of point] {\(\{cd\}\)};
            \node[left=3mm of a] {\(\{a\}\)};
            \node[above=3mm of plus] {\(\{a, cd\}\)};
        \end{tikzpicture}
        \caption{
            Représentation de l'expression régulière à l'aide d'un arbre syntaxique
        }\label{fig:arbre_syn}
    \end{figure}

    Comme on peut voir sur la figure~\ref{fig:arbre_syn} grâce à cette
    représentation, on peut calculer simplement le langage reconnu par l'expression
    régulière (ici représenté par les ensembles à côté de chaque arbre).
\end{example}

\vphantom{}

\begin{example}

    On comprendra aussi que l'expression \(E'(\mathbb{A}) = \varepsilon + b^* \cdot
    a\), dénote le langage \(L(E'(\mathbb{A})) = \{\varepsilon, b^* \cdot a\}
    \Leftrightarrow \{\varepsilon, a, b^+ \cdot a\}\).

    \begin{figure}[H]
        \centering
        \captionsetup{type=figure,justification=centering}
        \begin{tikzpicture}[
                mycircle/.style={
                        draw,
                        circle,
                        minimum height=.75cm,
                        minimum width=.75cm
                    },
                mysquare/.style={
                        draw,
                        rectangle,
                        minimum height=.5cm,
                        minimum width=.5cm,
                    }
            ]
            \node[mysquare] (plus) {+}
            child {node[mycircle] (epsilon) {\(\varepsilon\)}}
            child {
                    node[mysquare] (point) {\(\cdot \)}
                    child {
                            node[mysquare] (etoile) {*}
                            child {
                                    node[mycircle] (b) {b}
                                }
                        }
                    child {node[mycircle] (a) {a}}
                };

            \node[right=3mm of a] {\(\{a\}\)};
            \node[left=3mm of b] {\(\{b\}\)};
            \node[left=3mm of etoile] {\(\{b^*\}\)};
            \node[right=3mm of point] {\(\{b^* \cdot a\}\)};
            \node[left=3mm of epsilon] {\(\{\varepsilon\}\)};
            \node[above=3mm of plus] {\(\{\varepsilon, b^* \cdot a\}\)};
        \end{tikzpicture}
        \caption{
            Représentation de l'expression régulière à l'aide d'un arbre syntaxique
        }
    \end{figure}

\end{example}

\subsection{Fonction sur les \textit{ER}}

\textbullet~Une des fonctions les plus importantes sur les expressions régulières
est \(flnf\) permettant de calculer \(FLNF(E(\Sigma))\) qui est un tuple défini
comme ceci~:

\vphantom{}

\begin{center}
    \(FLNF(E(\Sigma)) = (F, L, \Theta, \delta)\)

    \begin{itemize}
        \item[\textbullet] \(F \subseteq \Sigma\)~: Ensemble des premiers
            symboles de l'expression régulière

            \vphantom{}

        \item[\textbullet] \(L \subseteq \Sigma\)~: Ensemble des derniers
            symboles de l'expression régulière

            \vphantom{}

        \item[\textbullet] \(\Theta\) =
            \(
            \begin{cases}
                \varepsilon, & \text{si}~ \varepsilon \in L(E(\Sigma)) \\
                \varnothing  & \text{sinon}
            \end{cases}
            \)

            \vphantom{}

        \item[\textbullet] \(\delta\)~: \(\Sigma \to S\) avec \(S \subseteq
            \Sigma\). \\
            Fonction renvoyant les symboles suivant du symbole passer en argument.
    \end{itemize}
\end{center}

La fonction \(flnf\) a donc comme signature~:

\begin{align}
    flnf: E(\Sigma) \to FLNF(E(\Sigma))
\end{align}

Et peut-être calculée de cette manière~:

\begin{align}
    flnf(\varepsilon) & = (\varnothing, \varnothing, \varepsilon, \delta) ~|~ \delta(a) = \varnothing, a \in \Sigma \\
    \vphantom{} \notag                                                                                              \\
    flnf(a)           & = (\{a\}, \{a\}, \varnothing, \delta) ~|~ \delta(a) = \varnothing, a
    \in \Sigma
\end{align}

\begin{gather}
    flnf(E(\Sigma) + G(\Sigma)) = (F \cup F', L \cup L', \Theta \cup \Theta', \delta'')~ \text{avec} \\
    \delta''(a) = \delta(a) \cup \delta'(a) ~|~ \forall a \in \Sigma \notag \\
    (F, L, \Theta, \delta) = flnf(E(\Sigma)) \land (F', L', \Theta', \delta') = flnf(G(\Sigma)) \notag
\end{gather}

\begin{gather}
    flnf(E(\Sigma) \cdot G(\Sigma)) = (F'', L'', \Theta \cap \Theta', \delta'')~ \text{avec} \\
    F'' = F \cup F' \cdot \Theta \notag \\
    L'' = L' \cup L \cdot \Theta' \notag \\
    \delta''(a) = \begin{cases} \delta(a) \cup \delta'(a) \cup F', & \text{si}~ a \in L \\ \delta(a) \cup \delta'(a) & \text{sinon}\end{cases} ~|~ \forall a \in \Sigma\notag \\
    (F, L, \Theta, \delta) = flnf(E(\Sigma)) \land (F', L', \Theta', \delta') = flnf(G(\Sigma)) \notag
\end{gather}

\begin{gather}
    flnf(E(\Sigma)^*) = (F, L, \varepsilon, \delta')~ \text{avec} \\
    \delta'(a) = \begin{cases} \delta(a) \cup F, & \text{si}~ a \in L \\ \delta(a) & \text{sinon}\end{cases} ~|~ \forall a \in \Sigma\notag \\
    (F, L, \Theta, \delta) = flnf(E(\Sigma)) \notag
\end{gather}

\begin{example}
    Prenons par exemple l'expression régulière suivante \(E(\mathbb{A}) = a \cdot b
    + c \cdot d\), avec \(\mathbb{A} = \{a, b, c, d\}\). Toujours à l'aide d'un
    arbre syntaxique, on peut calculer ce que \(flnf(E(\mathbb{A}))\) donnerait.

    \begin{figure}[H]
        \centering
        \captionsetup{type=figure,justification=centering}
        \begin{tikzpicture}[
                level 1/.style={
                        sibling distance=6cm
                    },
                level 2/.style={
                        sibling distance=3cm
                    },
                mycircle/.style={
                        draw,
                        circle,
                        minimum height=.75cm,
                        minimum width=.75cm
                    },
                mysquare/.style={
                        draw,
                        rectangle,
                        minimum height=.5cm,
                        minimum width=.5cm,
                    }
            ]
            \node[mycircle] (plus) {+}
            child {
                    node[mysquare] (point) {\(\cdot\)}
                    child {node[mycircle] (a) {a} }
                    child {node[mycircle] (b) {b} }
                }
            child {
                    node[mysquare] (point2) {\(\cdot\)}
                    child {node[mycircle] (c) {c} }
                    child {node[mycircle] (d) {d}}
                };

            \node[below=3mm of a] {\((\{a\}, \{a\}, \varnothing, \varnothing)\)};
            \node[below=3mm of b] {\((\{b\}, \{b\}, \varnothing, \varnothing)\)};
            \node[below=3mm of c] {\((\{c\}, \{c\}, \varnothing, \varnothing)\)};
            \node[below=3mm of d] {\((\{d\}, \{d\}, \varnothing, \varnothing)\)};

            \node[left=3mm of point] {\((\{a\}, \{b\}, \varnothing, \{(a, b)\})\)};
            \node[right=3mm of point2] {\((\{c\}, \{d\}, \varnothing, \{(c, d)\})\)};

            \node[above=3mm of plus] {\((\{a, c\}, \{b, d\}, \varnothing, \{(a, b),(c, d)\})\)};
        \end{tikzpicture}
        \caption{
            Représentation de l'expression régulière à l'aide d'un arbre syntaxique.
            Pour des raisons de compréhension \(\delta\) est représenté à l'aide
            d'un ensemble de couple.
        }
    \end{figure}

    Il advient que \(flnf(E(\mathbb{A})) = \{\{a, c\}, \{b, d\}, \varnothing,
    \delta\}\) avec \(\delta\) qui est défini comme ceci~:

    \begin{align*}
        \delta(a) & = \{b\}       \\
        \delta(b) & = \varnothing \\
        \delta(c) & = \{d\}       \\
        \delta(d) & = \varnothing
    \end{align*}
\end{example}

\vphantom{}

\begin{example}
    Un autre exemple pourrait être \(E'(\mathbb{A}) = (a + b) \cdot c^*\), avec cet
    exemple, on voit l'utilité de la parenthèse, car sans elle la concaténation
    aurait été sur \(b \cdot c^*\) et comme dit précédemment
    (\ref{text:parenthese}) son calcul reste le même que si c'était une union.

    \begin{figure}[H]
        \centering
        \captionsetup{type=figure,justification=centering}
        \begin{tikzpicture}[
                level 1/.style={
                        sibling distance=6cm
                    },
                level 2/.style={
                        sibling distance=3cm
                    },
                mycircle/.style={
                        draw,
                        circle,
                        minimum height=.75cm,
                        minimum width=.75cm
                    },
                mysquare/.style={
                        draw,
                        rectangle,
                        minimum height=.5cm,
                        minimum width=.5cm,
                    }
            ]
            \node[mycircle] (point) {\(\cdot\)}
            child {
                    node[mysquare] (plus) {\(+\)}
                    child {node[mycircle] (a) {a} }
                    child {node[mycircle] (b) {b} }
                }
            child {
                    node[mysquare] (etoile) {\(*\)}
                    child {node[mycircle] (c) {c} }
                };

            \node[below=3mm of a] {\((\{a\}, \{a\}, \varnothing, \varnothing)\)};
            \node[below=3mm of b] {\((\{b\}, \{b\}, \varnothing, \varnothing)\)};
            \node[below=3mm of c] {\((\{c\}, \{c\}, \varnothing, \varnothing)\)};

            \node[left=3mm of plus] {\((\{a, b\}, \{a, b\}, \varnothing, \varnothing)\)};
            \node[right=3mm of etoile] {\((\{c\}, \{c\}, \varepsilon, \{(c, c)\})\)};

            \node[above=3mm of point] {\((\{a, b\}, \{a, b, c\}, \varnothing, \{(a, c), (b, c), (c, c)\})\)};
        \end{tikzpicture}
        \caption{
            Représentation de l'expression régulière à l'aide d'un arbre syntaxique.
            Pour des raisons de compréhension \(\delta\) est représenté à l'aide
            d'un ensemble de couple.
        }
    \end{figure}

    Ce qui fait que \(flnf(E'(\mathbb{A})) = (\{a, b\}, \{a, b, c\}, \varnothing,
    \delta')\) avec \(\delta'\) qui est défini comme décrit après~:

    \begin{align*}
        \delta(a) & = \{c\}       \\
        \delta(b) & = \{c\}       \\
        \delta(c) & = \{c\}       \\
        \delta(d) & = \varnothing
    \end{align*}

\end{example}

\vphantom{}

\textbullet~Une autre fonction qui s'applique sur les expressions régulières
est \(linearization\);(elle peut paraitre inutile), mais elle nous servira dans
la Section~\ref{sec:glushkov}. Sa signature est~:

\begin{gather*}
    linearization: E(\Sigma) \times \mathbb{N} \to (E(\Sigma^{\mathbb{N}}), \mathbb{N}) \\
    \text{avec } \Sigma^{\mathbb{N}} = (\Sigma, \mathbb{N}) \land \forall ((a, b), (c, d)) \in (\Sigma^{\mathbb{N}})^2 ~|~ b = d \Rightarrow a = c \Rightarrow (a, b) = (c, d)
\end{gather*}

Elle peut être définie récursivement de cette manière~:

\begin{gather}
    linearization(\varepsilon, n) = (\varepsilon, n) \\
    linearization(a, n) = ((a, n), n + 1) \\
    linearization(E(\Sigma) + F(\Sigma), n) = (E' + F', n'') \quad \text{avec } \\
    (E', n') \leftarrow linearization(E(\Sigma), n) \notag \\
    (F', n'') \leftarrow linearization(F(\Sigma), n') \notag \\
    linearization(E(\Sigma) \cdot F(\Sigma), n) = (E' \cdot F', n'') \quad
    \text{avec } \\
    (E', n') \leftarrow linearization(E(\Sigma), n) \notag \\
    (F', n'') \leftarrow linearization(F(\Sigma), n') \notag \\
    linearization(E(\Sigma)^*, n) = (E'^*, n') \quad \text{avec} \\
    (E', n') \leftarrow linearization(E(\Sigma), n) \notag
\end{gather}

Avec cette définition, on peut voir que tous les deuxièmes éléments du couple
des symboles du résultat \(E'\) avec \(linearization(E(\Sigma), n) = (E', m)\),
seront supérieurs ou égaux à \(n\) et inférieurs stricts à \(m\).

\begin{example}

    Si on reprend cette expression régulière \(E(\mathbb{A}) = \varepsilon + b^*
    \cdot a\), avec \(\mathbb{A} = \{a, b, c, d\}\).

    \begin{figure}[H]
        \centering
        \captionsetup{type=figure,justification=centering}
        \begin{tikzpicture}[
                mycircle/.style={
                        draw,
                        rectangle,
                        rounded corners=.375cm,
                        minimum height=.75cm,
                        minimum width=.75cm
                    },
                mysquare/.style={
                        draw,
                        rectangle,
                        minimum height=.5cm,
                        minimum width=.5cm,
                    }
            ]
            \node[mysquare] (1) {+}
            child {node[mycircle] {\(\varepsilon\)}}
            child {
                    node[mysquare] {\(\cdot \)}
                    child {
                            node[mysquare] {*}
                            child {
                                    node[mycircle] {b}
                                }
                        }
                    child {node[mycircle] {a}}
                };
            \node[mysquare, right=7cm of 1] (2) {+}
            child {node[mycircle] {\(\varepsilon\)}}
            child {
                    node[mysquare] {\(\cdot \)}
                    child {
                            node[mysquare] {*}
                            child {
                                    node[mycircle] {(b, 1)}
                                }
                        }
                    child {node[mycircle] {(a, 2)}}
                };

            \draw[line width=.5mm, -{Stealth[length=5mm, open]}] ($(1.east) + (1.5cm, -2cm)$) -- node[midway, above=2mm] {\(linearization\)} ($(2.west) + (-1.5cm, -2cm)$);
        \end{tikzpicture}
        \caption{
            Représentation à l'aide d'un arbre syntaxique de l'expression régulière
            une fois après avoir fait appel à \(linearization\) sur elle.
        }
    \end{figure}

\end{example}

\vphantom{}

\subsection{Conclusion}

On saisit aisément que ces expressions ont beau être simples (peu d'opération
comparé aux expressions régulières d'\textit{UNIX}). Elles n'en sont rien pas
complètes, on peut voir qu'elles permettent de décrire des langages très
complexes et en quantité infinie. Malheureusement, il est difficile de savoir
si un mot est reconnu par une expression régulière simplement. Par exemple
est-ce-que le mot \(eipipipipipip\) est reconnu par cette expression \(((((o
\cdot \varepsilon)+(\varepsilon \cdot e))+((g\cdot \varepsilon) \cdot
\varepsilon^*)) \cdot ((\varepsilon \cdot i)\cdot (p+\varepsilon))^*)\)~? La
réponse est oui. C'est pour cela qu'il serait peut-être intéressant d'utiliser
une autre structure de donnée pour reconnaitre des mots, comme les automates
que nous allons voir maintenant.