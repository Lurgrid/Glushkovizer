\section{Conclusion}

Ce document a fourni une définition formelle des concepts clefs utilisés dans
la théorie des langages et les automates. Nous avons d'abord introduit les
notions de mots, de langages et les opérations fondamentales qui leur sont
associées. Ensuite, nous avons exploré les expressions régulières, leurs
définitions et les fonctions qui peuvent être appliquées sur elles. Par la
suite, nous avons étudié les automates, en particulier ceux sans transitions
\(\varepsilon\), et les fonctions qui leur sont appliquées. Enfin, nous avons
abordé les automates de Glushkov, décrivant leur construction et leurs
propriétés.

\vphantom{}

Bien que nous n'ayons couvert qu'une partie des concepts et des opérations
possibles, cette introduction vise à fournir une base solide pour comprendre et
utiliser ces outils puissants. Pour approfondir vos connaissances, nous vous
encourageons à consulter des ressources supplémentaires en théorie des langages
et des automates.
