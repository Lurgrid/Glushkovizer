\documentclass[12pt]{article}
\usepackage[french]{babel}
\usepackage{cmpt}
\usepackage{amsfonts}
\usepackage{amssymb}
\usepackage{amsthm}
\usepackage{tikz}
\usepackage{float}
\usepackage{enumitem}
\usepackage{caption}
\usepackage{csquotes}
\usepackage[
    backend=biber,
    style=alphabetic,
    sorting=ynt
]{biblatex}
\addbibresource{references.bib}
\nocite{*}
\usetikzlibrary{positioning}
\usetikzlibrary{arrows.meta}
\usetikzlibrary{automata}
\usetikzlibrary{arrows}
\usetikzlibrary{shapes.geometric}
\usetikzlibrary{calc}

\theoremstyle{definition}

\newtheorem{definition}{Définition}[section]
\newtheorem{remark}[definition]{Remarque}
\newtheorem{lemma}[definition]{Lemme}
\newtheorem{example}{Example}[definition]

\makeatletter
\renewcommand{\thedefinition}{\arabic{section}.\arabic{definition}}
\makeatother

%--- begin document -----------------------------------------------------------

\title{Définition Formelle --- Glushkovizer}
\author{}
\date{}

\begin{document}

\maketitle

\begin{abstract}
    Ce document constitue la définition formelle des différents types de
    données utilisés tout au long de cette librairie. Dans un premier temps,
    nous nous concentrerons sur les expressions régulières et les fonctions
    définies sur celles-ci. Puis, dans un second temps, nous nous intéresserons
    aux automates et à leurs diverses fonctions définies sur eux. Enfin, pour
    finir, nous parlerons d'automates particuliers, ceux de Glushkov, nous
    aborderons leurs constructions ainsi que leurs propriétés.
\end{abstract}

\newpage

\tableofcontents

\section{Prélude}

Pour la compréhension de l'ensemble de ce document, nous avons besoin de
plusieurs notions de théorie des langages. C'est donc pourquoi dans cette
partie, nous allons étudier les différentes notions nécessaires. Dans un
premier temps, nous allons définir ce qu'est un mot et quelles sont les
opérations sur les mots. Enfin, dans un second temps, nous définirons ce qu'est
un langage et quelles opérations sont munies par les langages.

\subsection{Les mots}

\begin{definition}
    Un \textit{alphabet} \(\Sigma\) est un ensemble fini non-vide de symboles.
    Un \textit{mot} est une suite finie de symboles sur un alphabet \(\Sigma\).
    Le mot composé de zéro symbole est appelé mot vide et est noté
    \(\varepsilon\).
\end{definition}

\begin{example}
    \begin{gather*}
        \Sigma = \{a, b, c, d\} \\
        w = abbcdda
    \end{gather*}
\end{example}

\begin{definition}
    On parlera de la \textit{longueur d'un mot} \(w\) noté \(|w|\) pour
    désigner le nombre de symboles qui le composent. % De même, on notera \({|w|}_a\) pour parler du nombre de \(a\) dans le mot% \(w\).
\end{definition}

\begin{example}
    \begin{gather*}
        w = abbcdda \notag \\
        |w| = 7 \\
        % {|w|}_d = 2
    \end{gather*}
\end{example}

\begin{definition}
    Une des opérations sur les mots est la concaténation de mots. On notera la
    \textit{concaténation} de deux mots \(u = a_1 \cdots a_n\) et \(v = b_1
    \cdots b_n\) par \(u \cdot v\). Qui est ainsi égal à \(u \cdot v = a_1
    \cdots a_n b_1 \cdots b_n\). On définit l'ensemble des mots sur \(\Sigma\)
    par \(\Sigma ^ *\). On notera que~:

    \begin{itemize}[label=\textbullet]
        \item La concaténation est associative \((w \cdot u) \cdot v = w \cdot
              (u \cdot v)\).
        \item La concaténation admet un élément neutre \(u \cdot \varepsilon =
              \varepsilon \cdot u = u\).
    \end{itemize}

    \noindent Ce qui implique que l'ensemble \(\Sigma ^ *\) muni de la concaténation
    \((\Sigma, \cdot)\) forme un monoïde.
\end{definition}

\begin{example}
    \begin{gather*}
        u = abab \\
        v = cdcd \\
        u \cdot v = ababcdcd
    \end{gather*}
\end{example}

% \begin{definition}
%     Grâce à cette opération sur les mots, on peut définir ce qu'est un
%     \textit{facteur}. Un facteur \(u\) d'un mot \(w\) est une suite extraite de la
%     suite de lettre qui composent le mot \(w\). Autrement dit \(u\) est un sous mot
%     de \(w\) si \(\exists (v, x) \in (\Sigma ^ *)^2 ~|~ w = v \cdot u \cdot x\), de
%     plus~:

%     \begin{itemize}[label=\textbullet]
%         \item On parlera de \textit{préfixe} quand \(v = \varepsilon\).
%         \item On parlera de \textit{suffixe} quand \(x = \varepsilon\).
%         \item Enfin, on parlera de \textit{facteur propre} quand \(v \neq \varepsilon \land x
%               \neq \varepsilon\).
%     \end{itemize}

%     \noindent On remarquera que \(\varepsilon\) est~: \textit{préfixe}, \textit{suffixe} et
%     \textit{facteur} de tout mot.
% \end{definition}

% \begin{example}
%     \begin{gather*}
%         w = abbcdda \\
%         u = bcd \\
%         y = abb \\
%         w = ab \cdot u \cdot da\\
%         u \text{ est donc un facteur propre} \notag \\
%         w = y \cdot cdda \quad \\
%         y \text{ est un facteur et un préfixe} \notag
%     \end{gather*}
% \end{example}

% \begin{definition}
%     De plus aussi grâce à la concaténation, nous pouvons définir le \textit{miroir}
%     d'un mot \(w\) noté \(\overleftarrow{w}\). Qui est ainsi défini récursivement
%     comme ceci~:

%     \begin{gather*}
%         \overleftarrow{\varepsilon} = \varepsilon \\
%         \overleftarrow{a} = a \\
%         \overleftarrow{u \cdot a} = a \cdot \overleftarrow{u} \\
%         \text{Avec } a \in \Sigma \text{ et } u \in \Sigma ^ * \notag
%     \end{gather*}

%     \noindent On remarquera qu'un mot est un palindrome si \(u = \overleftarrow{u}\).
% \end{definition}

% \begin{example}
%     \begin{gather*}
%         w = abbcdda \\
%         \overleftarrow{w} = addcbba
%     \end{gather*}
% \end{example}

\subsection{Les langages}

\begin{definition}
    Un \textit{langage} \(L\) est un ensemble de mots sur un alphabet fini
    \(\Sigma\). On appellera \textit{langage vide} le langage ne comportant
    aucun mot, ainsi défini comme ceci~: \(L = \varnothing\).
\end{definition}

\begin{example}
    \begin{gather*}
        \Sigma = \{a, b, c, d\} \\
        L_1 = \{a, aa, bc, da, \varepsilon\} \\
        L_2 = \varnothing
    \end{gather*}
\end{example}

% \begin{definition}
%     L'équivalant de la longueur d'un mot sur les langages est donc le
%     \textit{cardinal} du langage. On remarquera qu'un langage n'est pas forcément
%     fini et qu'alors son \textit{cardinal} peut être infini.
% \end{definition}

% \begin{example}
%     \begin{gather*}
%         \Sigma = \{a, b, c, d\} \\
%         L_1 = \{a, aa, bc, da, \varepsilon\} \\
%         L_2 = \varnothing \\
%         | L_1 | = 5 \\
%         | L_2 | = 0
%     \end{gather*}
% \end{example}

\begin{definition}
    L'une des opérations sur les langages est l'\textit{union}. On parlera de
    l'union de langage notée \(L_1 \cup L_2\) et définie comme ceci~:

    \begin{gather*}
        L_1 \cup L_2 = \{w \in \Sigma ^ * ~|~ w \in L_1 \lor w \in L_2\}
    \end{gather*}

    \noindent On notera que l'union est associative, commutative et admet un
    élément neutre (\(\varnothing\)).
\end{definition}

% \begin{definition}
%     À l'image de l'union de langages, nous avons aussi l'\textit{intersection} de
%     langage noté \(L_1 \cap L_2\) et donc défini comme cela~:

%     \begin{gather*}
%         L_1 \cap L_2 = \{w \in \Sigma ^ * ~|~ w \in L_1 \land w \in L_2\}
%     \end{gather*}

%     \noindent On notera qu'elle aussi est associative, commutative et admet
%     aussi un élément neutre (\(\Sigma ^ *\)). On remarquera que \(\varnothing\) est
%     un élément absorbant.
% \end{definition}

\begin{example}
    \begin{gather*}
        \Sigma = \{a, b, c, d\} \\
        L_1 = \{\varepsilon, a, aa, bc, da\} \\
        L_2 = \{d, aa, cd\} \\
        L_1 \cup L_2 = \{\varepsilon, a, d, aa, cd, bc, da\} \\
        % L_1 \cap L_2 = \{aa\}
    \end{gather*}
\end{example}

\begin{definition}
    Une autre opération sur les langages est la \textit{concaténation}. Elle
    est définie en utilisant la concaténation des mots qui composent les
    langages. Cette opération est ainsi définie comme ceci~:

    \begin{gather*}
        L_1 \cdot L_2 = \{u \cdot v ~|~ u \in L_1, v \in L_2\}
    \end{gather*}

    \noindent On remarquera qu'elle est associative, pas commutative et admet un
    élément neutre (\(\{\varepsilon\}\)). Et que \(\varnothing\) est aussi un
    élément absorbant pour cette opération.
\end{definition}

\begin{example}
    \begin{gather*}
        \Sigma = \{a, b, c, d\} \\
        L_1 = \{\varepsilon, a, aa\} \\
        L_2 = \{d, cc\} \\
        L_1 \cdot L_2 = \{d, ad, aad, cc, acc, aacc\}
    \end{gather*}
\end{example}

\begin{definition}
    Par extension, on définit la \textit{copie n-ième} d'un langage \(L\) notée
    \(L^n\) et définit récursivement comme ceci~:

    \begin{gather*}
        L^0 = \{\varepsilon\} \\
        L^n = L^{n - 1} \cdot L
    \end{gather*}

    \noindent On remarquera que \(\varnothing^0 = \{\varepsilon\}\).
\end{definition}

\begin{example}
    \begin{gather*}
        \Sigma = \{a, b\} \\
        L = \{\varepsilon, a\} \\
        L^3 = \{\varepsilon, a, aa, aaa\}
    \end{gather*}
\end{example}

\begin{definition}
    Grâce à cette opération, on peut définir l'\textit{étoile} d'un langage
    notée \(L^*\). Qui peut être définie comme ceci~:

    \begin{gather*}
        L^* = \bigcup_{i \geq 0} L^i
    \end{gather*}
\end{definition}

\begin{example}
    \begin{gather*}
        L = \{a\} \\
        L^* = \{\varepsilon\} \cup \{a\} \cup \{aa\} \cup \cdots
    \end{gather*}
\end{example}

\subsection{Conclusion}

Nous avons défini les concepts de \textit{mot}, de \textit{langage} et
l'ensemble des opérations applicables à ces objets. Bien que nous n'ayons
couvert qu'une partie des opérations possibles, nous vous encourageons à
consulter un cours de théorie des langages pour obtenir des informations plus
détaillées. Nous vous conseillons ces
ressources~:~\cite{Harrison1978},~\cite{Autebert1994} et~\cite{Hopcroft2007}.


\section{Les expressions régulières}

Dans cette section, nous parlerons d'expressions régulières (\textit{ER}). Nous
allons nous concentrer sur un type bien particulier d'expressions régulières
qui ne seront pas les expressions régulières que nous pouvons voir plus
quotidiennement dans le domaine de l'informatique, les expressions régulières
\textit{UNIX}. Mais plut\^{o}t une version plus simple de celles-ci.

\subsection{Définition}

Nous allons noter une expression régulière \(E \in Exp(\Sigma)\), c'est-à-dire
une expression régulière où les symboles sont inclus dans l'ensemble \(\Sigma\)
et où \(Exp(\Sigma)\) représente l'ensemble des expressions sur \(\Sigma\).
Cette expression reconnait un langage qu'on pourra appeler \(L(E)\). Nous
pouvons définir une expression régulière récursivement de cette manière~:

\begin{align}
    E & = \varepsilon                 \\
    E & = a                           \\
    E & = F + G                       \\
    E & = F \cdot G                   \\
    E & = F^*                         \\
    E & = (F)\label{align:parenthese} \\
    \makebox[0pt][c]{
        \phantom{E } \text{
            avec \(E\), \(F\) et \(G\) des expressions régulières sur \(\Sigma\) et \(a\)
            un symbole de \(\Sigma\)
        }
    } \notag
\end{align}

On notera que \(*\) est prioritaire sur \(\cdot\) qui est lui-même prioritaire
sur \(+\) et qu'ils sont tous deux associatifs à gauche. On comprend donc
pourquoi l'équation~(\ref{align:parenthese}) existe, elle est là pour des
raisons de priorité. Il est alors évident de calculer les diverses fonctions
sur celle-ci, c'est pour cela qu'on ne précisera pas son
calcul\label{subsec:parenthese}. On peut définir chaque équation comme ceci~:

\vphantom{}

\begin{itemize}
    \item[\textbullet] \textbf{\(E = \varepsilon\)~:}
        Représente le mot vide, de ce fait un mot de longueur zéro. Il peut être
        parfois représenté par \og{}\$\fg{}.

        \vphantom{}

    \item[\textbullet] \textbf{\(E = a\)~:} Représente un symbole
        présent dans l'ensemble \(\Sigma\)

        \vphantom{}

    \item[\textbullet] \textbf{\(E = F + G\)~:} Représente l'union
        des deux expressions régulières \(F\) et \(G\). Par abus de langage, on
        peut aussi dire \(F\) \og{}ou\fg{} \(G\) pour représenter cette union.

        \vphantom{}

    \item[\textbullet] \textbf{\(E = F \cdot G\)~:}
        Représente la concaténation des deux expressions régulières \(F\) et
        \(G\).

        \vphantom{}

    \item[\textbullet] \textbf{\(E = F^* \)~:} Représente l'union infinie de
        copie de \(F\), cette répétition incluant la puissance zéro et donc le
        mot vide.
\end{itemize}

\vphantom{}

Pour calculer le langage que dénote l'expression régulière, on peut le calculer
récursivement de cette manière~:

\begin{align*}
    L(\varepsilon) & = \{\varepsilon\} \\
    L(a) & = \{a\}           \\
    L(F + G) & = L(F) \cup L(G)  \\
    L(F \cdot G) & = L(F) \cdot L(G) \\
    L(F^*) & = (L(F))^*        \\
    \makebox[0pt][c]{
        \phantom{E } \text{
            avec \(F\) et \(G\) des expressions régulières sur \(\Sigma\) et \(a\)
            un symbole de \(\Sigma\)
        }
    } \notag
\end{align*}

\begin{example}
    On comprendra ainsi que l'expression \(E = a+c \cdot d\) avec \(E \in
    Exp(\Sigma)\) et \(\Sigma = \{a, b, c, d\}\), dénote le langage \(L(E) =
    \{a, cd\}\). Car on peut représenter \(E\) comme ceci~:

    \begin{figure}[H]
        \centering
        \captionsetup{type=figure,justification=centering}
        \begin{tikzpicture}[
                mycircle/.style={
                        draw,
                        circle,
                        minimum height=.75cm,
                        minimum width=.75cm
                    },
                mysquare/.style={
                        draw,
                        rectangle,
                        minimum height=.5cm,
                        minimum width=.5cm,
                    }
            ]
            \node[mysquare] (plus) {+}
            child {node[mycircle] (a) {a}}
            child {
                    node[mysquare] (point) {.}
                    child {node[mycircle] (c) {c}}
                    child {node[mycircle] (d) {d}}
                };

            \node[right=3mm of d] {\(\{d\}\)};
            \node[left=3mm of c] {\(\{c\}\)};
            \node[right=3mm of point] {\(\{cd\}\)};
            \node[left=3mm of a] {\(\{a\}\)};
            \node[above=3mm of plus] {\(\{a, cd\}\)};
        \end{tikzpicture}
        \caption{
            Représentation de l'expression régulière à l'aide d'un arbre syntaxique
        }\label{fig:arbre_syn}
    \end{figure}

    Comme on peut voir sur la Figure~\ref{fig:arbre_syn}, grâce à cette
    représentation, on peut calculer simplement le langage reconnu par
    l'expression régulière (ici représenté par les ensembles à c\^{o}té de
    chaque arbre).
\end{example}

% \vphantom{}

% \begin{example}

%     On comprendra aussi que l'expression \(E' = \varepsilon + b^* \cdot a\), dénote
%     le langage \(L(E') = \{\varepsilon, b^* \cdot a\}\).

%     \begin{figure}[H]
%         \centering
%         \captionsetup{type=figure,justification=centering}
%         \begin{tikzpicture}[
%                 mycircle/.style={
%                         draw,
%                         circle,
%                         minimum height=.75cm,
%                         minimum width=.75cm
%                     },
%                 mysquare/.style={
%                         draw,
%                         rectangle,
%                         minimum height=.5cm,
%                         minimum width=.5cm,
%                     }
%             ]
%             \node[mysquare] (plus) {+}
%             child {node[mycircle] (epsilon) {\(\varepsilon\)}}
%             child {
%                     node[mysquare] (point) {\(\cdot \)}
%                     child {
%                             node[mysquare] (etoile) {*}
%                             child {
%                                     node[mycircle] (b) {b}
%                                 }
%                         }
%                     child {node[mycircle] (a) {a}}
%                 };

%             \node[right=3mm of a] {\(\{a\}\)};
%             \node[left=3mm of b] {\(\{b\}\)};
%             \node[left=3mm of etoile] {\(\{b^*\}\)};
%             \node[right=3mm of point] {\(\{b^* \cdot a\}\)};
%             \node[left=3mm of epsilon] {\(\{\varepsilon\}\)};
%             \node[above=3mm of plus] {\(\{\varepsilon, b^* \cdot a\}\)};
%         \end{tikzpicture}
%         \caption{
%             Représentation de l'expression régulière à l'aide d'un arbre syntaxique
%         }
%     \end{figure}

% \end{example}

\subsection{Fonction sur les \textit{ER}}

Plusieurs informations sur les expressions régulières nous seront utiles, comme
l'ensemble des premiers/derniers symboles des mots du langage décrit par
l'expression. Il serait aussi intéressant de savoir si son langage contient le
mot vide. Et d'avoir les successeurs des symboles, c'est-à-dire les symboles
suivant un symbole donné.

\vphantom{}

On pourrait calculer individuellement chaque information, mais nous pouvons
calculer tout d'un coup avec une fonction qu'on pourrait appeler \(flnf\). Elle
permet de calculer un tuple contenant toutes ces informations pour une
expression régulière donné.

\vphantom{}

On aurait donc pour une expression régulière \(E\) sur l'alphabet \(\Sigma\),
ceci~:

\begin{center}
    \(flnf(E) = (F, L, \Theta, \delta)\)

    \begin{itemize}
        \item[\textbullet] \(F \subseteq \Sigma\)~: Ensemble des premiers
            symboles de l'expression régulière

            \vphantom{}

        \item[\textbullet] \(L \subseteq \Sigma\)~: Ensemble des derniers
            symboles de l'expression régulière

            \vphantom{}

        \item[\textbullet] \(\Theta\) =
            \(
            \begin{cases}
                \{ \varepsilon \}, & \text{si } \varepsilon \in L(E) \\
                \varnothing        & \text{sinon}
            \end{cases}
            \)

            \vphantom{}

        \item[\textbullet] \(\delta\)~: \(\Sigma \to 2^{\Sigma}\) fonction
            renvoyant les successeurs du symbole donné
    \end{itemize}
\end{center}

La fonction \(flnf\) a donc comme signature~:

\begin{align*}
    flnf: Exp(\Sigma) \to (2^{\Sigma} \times 2^{\Sigma} \times
    \{\varnothing,\{\varepsilon\}\} \times \Sigma \to 2^{\Sigma})
\end{align*}

Et peut-être calculée de cette manière, pour \(E\) et \(G\) des expressions
régulière sur l'alphabet \(\Sigma\) et \(a\) un symbole de \(\Sigma\)~:

\begin{align*}
    flnf(\varepsilon) & = (\varnothing, \varnothing, \varepsilon, \delta) ~|~
    \delta(a) = \varnothing, a \in \Sigma                                     \\
    \vphantom{} \notag                                                        \\
    flnf(a) & = (\{a\}, \{a\}, \varnothing, \delta) ~|~ \delta(a) =
    \varnothing, a \in \Sigma
\end{align*}

\begin{gather*}
    flnf(E + G) = (F \cup F', L \cup L', \Theta \cup \Theta', \delta'')~
    \text{avec} \\
    \delta''(a) = \delta(a) \cup \delta'(a) ~|~ \forall a \in \Sigma \notag \\
    (F, L, \Theta, \delta) = flnf(E) \land (F', L', \Theta', \delta') = flnf(G) \notag
\end{gather*}

\begin{gather*}
    flnf(E \cdot G) = (F'', L'', \Theta \cap \Theta', \delta'')~ \text{avec} \\
    F'' = F \cup F' \cdot \Theta \notag \\
    L'' = L' \cup L \cdot \Theta' \notag \\
    \delta''(a) = \begin{cases} \delta(a) \cup \delta'(a) \cup F', & \text{si}~ a \in L \\ \delta(a) \cup \delta'(a) & \text{sinon}\end{cases} ~|~ \forall a \in \Sigma\notag \\
    (F, L, \Theta, \delta) = flnf(E) \land (F', L', \Theta', \delta') = flnf(G) \notag
\end{gather*}

\begin{gather*}
    flnf(E^*) = (F, L, \{\varepsilon\}, \delta')~ \text{avec} \\
    \delta'(a) = \begin{cases} \delta(a) \cup F, & \text{si}~ a \in L \\ \delta(a) & \text{sinon}\end{cases} ~|~ \forall a \in \Sigma\notag \\
    (F, L, \Theta, \delta) = flnf(E) \notag
\end{gather*}

\begin{remark}
    Il existe un isomorphisme entre les fonctions et les couple antécédents,
    images. Ce qui fait que la fonction des successeurs pourra être représenté
    à l'aide d'un couple.
\end{remark}

\begin{example}
    Prenons par exemple l'expression régulière suivante \(E = a \cdot b + c
    \cdot d\), avec \(E \in Exp(\Sigma)\) et \(\Sigma = \{a, b, c, d\}\).
    Toujours à l'aide d'un arbre syntaxique, on peut calculer ce que
    \(flnf(E)\) donnerait.

    \begin{figure}[H]
        \centering
        \captionsetup{type=figure,justification=centering}
        \begin{tikzpicture}[
                level 1/.style={
                        sibling distance=6cm
                    },
                level 2/.style={
                        sibling distance=3cm
                    },
                mycircle/.style={
                        draw,
                        circle,
                        minimum height=.75cm,
                        minimum width=.75cm
                    },
                mysquare/.style={
                        draw,
                        rectangle,
                        minimum height=.5cm,
                        minimum width=.5cm,
                    }
            ]
            \node[mycircle] (plus) {+}
            child {
                    node[mysquare] (point) {\(\cdot\)}
                    child {node[mycircle] (a) {a} }
                    child {node[mycircle] (b) {b} }
                }
            child {
                    node[mysquare] (point2) {\(\cdot\)}
                    child {node[mycircle] (c) {c} }
                    child {node[mycircle] (d) {d}}
                };

            \node[below=3mm of a] {\((\{a\}, \{a\}, \varnothing, \varnothing)\)};
            \node[below=3mm of b] {\((\{b\}, \{b\}, \varnothing, \varnothing)\)};
            \node[below=3mm of c] {\((\{c\}, \{c\}, \varnothing, \varnothing)\)};
            \node[below=3mm of d] {\((\{d\}, \{d\}, \varnothing, \varnothing)\)};

            \node[left=3mm of point] {\((\{a\}, \{b\}, \varnothing, \{(a, b)\})\)};
            \node[right=3mm of point2] {\((\{c\}, \{d\}, \varnothing, \{(c, d)\})\)};

            \node[above=3mm of plus] {\((\{a, c\}, \{b, d\}, \varnothing, \{(a, b),(c, d)\})\)};
        \end{tikzpicture}
        \caption{
            Représentation de l'expression régulière à l'aide d'un arbre
            syntaxique.
        }
    \end{figure}

    Il advient que \(flnf(E) = \{\{a, c\}, \{b, d\}, \varnothing, \delta\}\)
    avec \(\delta\) qui est défini comme ceci~:

    \begin{align*}
        \delta(a) & = \{b\}       \\
        \delta(b) & = \varnothing \\
        \delta(c) & = \{d\}       \\
        \delta(d) & = \varnothing
    \end{align*}
\end{example}

\vphantom{}

\begin{example}
    Un autre exemple pourrait être \(E' = (a + b) \cdot c^*\), avec cet
    exemple, on voit l'utilité de la parenthèse, car sans elle la concaténation
    aurait été sur \(b \cdot c^*\). Et comme dit précédemment
    (\ref{subsec:parenthese}), son calcul revient à calculer l'expression
    contenue entre les parenthèses.

    \begin{figure}[H]
        \centering
        \captionsetup{type=figure,justification=centering}
        \begin{tikzpicture}[
                level 1/.style={
                        sibling distance=6cm
                    },
                level 2/.style={
                        sibling distance=3cm
                    },
                mycircle/.style={
                        draw,
                        circle,
                        minimum height=.75cm,
                        minimum width=.75cm
                    },
                mysquare/.style={
                        draw,
                        rectangle,
                        minimum height=.5cm,
                        minimum width=.5cm,
                    }
            ]
            \node[mycircle] (point) {\(\cdot\)}
            child {
                    node[mysquare] (plus) {\(+\)}
                    child {node[mycircle] (a) {a} }
                    child {node[mycircle] (b) {b} }
                }
            child {
                    node[mysquare] (etoile) {\(*\)}
                    child {node[mycircle] (c) {c} }
                };

            \node[below=3mm of a] {\((\{a\}, \{a\}, \varnothing, \varnothing)\)};
            \node[below=3mm of b] {\((\{b\}, \{b\}, \varnothing, \varnothing)\)};
            \node[below=3mm of c] {\((\{c\}, \{c\}, \varnothing, \varnothing)\)};

            \node[left=3mm of plus] {\((\{a, b\}, \{a, b\}, \varnothing, \varnothing)\)};
            \node[right=3mm of etoile] {\((\{c\}, \{c\}, \varepsilon, \{(c, c)\})\)};

            \node[above=3mm of point] {\((\{a, b\}, \{a, b, c\}, \varnothing, \{(a, c), (b, c), (c, c)\})\)};
        \end{tikzpicture}
        \caption{
            Représentation de l'expression régulière à l'aide d'un arbre
            syntaxique.
        }
    \end{figure}

    Ce qui fait que \(flnf(E') = (\{a, b\}, \{a, b, c\}, \varnothing,
    \delta')\) avec \(\delta'\) qui est défini comme décrit après~:

    \begin{align*}
        \delta'(a) & = \{c\}       \\
        \delta'(b) & = \{c\}       \\
        \delta'(c) & = \{c\}       \\
        \delta'(d) & = \varnothing
    \end{align*}

\end{example}

\vphantom{}

Une autre fonction qui s'applique aux expressions régulières est
\(linearization\)~; (elle peut paraitre inutile, mais) elle nous servira dans
la Section~\ref{sec:glushkov}. Sa signature est~:

\begin{gather*}
    linearization: Exp(\Sigma) \to Exp(\Sigma \times \mathbb{N}) \\
\end{gather*}

Elle peut être définie de cette manière, pour \(a \in \Sigma\) et \((E, F) \in
(Exp(\Sigma))^2\)~:

\begin{gather*}
    linearization(E) = \pi_2(linearization\_aux(E, 1)) \quad \text{avec}
\end{gather*}

\noindent Avec \(\pi_n\) la fonction de projection sur les tuples et
\(linearization\_aux\) définie récursivement comme ceci~:

\begin{gather*}
    linearization\_aux(\varepsilon, n) = (\varepsilon, n) \\
    linearization\_aux(a, n) = ((a, n), n + 1) \\
    linearization\_aux(E + F, n) = (E' + F', n'') \quad \text{avec} \\
    (E', n') \leftarrow linearization\_aux(E, n) \notag \\
    (F', n'') \leftarrow linearization\_aux(F, n') \notag \\
    linearization\_aux(E \cdot F, n) = (E' \cdot F', n'') \quad \text{avec} \\
    (E', n') \leftarrow linearization\_aux(E, n) \notag \\
    (F', n'') \leftarrow linearization\_aux(F, n') \notag \\
    linearization\_aux(E^*, n) = (E'^*, n') \quad \text{avec} \\
    (E', n') \leftarrow linearization\_aux(E, n) \notag
\end{gather*}

Avec cette définition, on peut voir que tous les symboles sont associés à un
unique entier. Ce qui fait que l'expression régulière résultante ne contient
que des symboles uniques. Et que, de ce fait, si deux couples partagent le même
entier, cela implique qu'ils ont la même valeur de symbole.

\begin{example}

    Si on prend l'expression régulière \(E = \varepsilon + b^* \cdot b\), avec
    \(E \in Exp(\Sigma)\) et \(\Sigma = \{a, b, c, d\}\).

    \begin{figure}[H]
        \centering
        \captionsetup{type=figure,justification=centering}
        \begin{tikzpicture}[
                mycircle/.style={
                        draw,
                        rectangle,
                        rounded corners=.375cm,
                        minimum height=.75cm,
                        minimum width=.75cm
                    },
                mysquare/.style={
                        draw,
                        rectangle,
                        minimum height=.5cm,
                        minimum width=.5cm,
                    }
            ]
            \node[mysquare] (1) {+}
            child {node[mycircle] {\(\varepsilon\)}}
            child {
                    node[mysquare] {\(\cdot \)}
                    child {
                            node[mysquare] {*}
                            child {
                                    node[mycircle] {b}
                                }
                        }
                    child {node[mycircle] {b}}
                };
            \node[mysquare, right=7cm of 1] (2) {+}
            child {node[mycircle] {\(\varepsilon\)}}
            child {
                    node[mysquare] {\(\cdot \)}
                    child {
                            node[mysquare] {*}
                            child {
                                    node[mycircle] {(b, 1)}
                                }
                        }
                    child {node[mycircle] {(b, 2)}}
                };

            \draw[line width=.5mm, -{Stealth[length=5mm, open]}] ($(1.east) + (1.5cm, -2cm)$) -- node[midway, above=2mm] {\(linearization\)} ($(2.west) + (-1.5cm, -2cm)$);
        \end{tikzpicture}
        \caption{
            Représentation à l'aide d'un arbre syntaxique de l'expression régulière
            une fois après avoir fait appel à \(linearization\) sur elle.
        }
    \end{figure}

\end{example}

\vphantom{}

\subsection{Conclusion}

On saisit aisément que ces expressions ont beau être simples (peu d'opération
comparé aux expressions régulières d'\textit{UNIX}). On peut voir qu'elles
permettent de décrire des langages très complexes et en quantité infinie. En
revanche, il est difficile de savoir si un mot est reconnu par une expression
régulière simplement. Par exemple est-ce que le mot \(eipipipipipip\) est
reconnu par cette expression \(((((o \cdot \varepsilon)+(\varepsilon \cdot
e))+((g\cdot \varepsilon) \cdot \varepsilon^*)) \cdot ((\varepsilon \cdot
i)\cdot (p+\varepsilon))^*)\)~? La réponse est oui. C'est pour cela qu'il
serait peut-être intéressant d'utiliser un autre objet pour reconnaitre des
mots, comme les automates que nous allons voir maintenant.


\section{Les automates}

Dans cette partie, nous parlerons des automates et plus particulièrement, nous
allons parler des automates sans \(\varepsilon\)-transition (des automates
utilisent des \(\varepsilon\)-transitions, comme ceux de
\textit{Thompson}~\cite{thompson1968programming}, qui sont utilisés par nos
ordinateurs). Pour autant, les automates que nous verrons ne sont pas limités
par le manque de ces transitions.

\subsection{Définition}

Comme dit précédemment, un automate est un objet mathématique reconnaissant un
langage. On notera \(M \in AFN(\Sigma, \eta)\) l'automate qui a pour transition
des valeurs dans \(\Sigma\), des valeurs \og{}d'état\fg{} dans \(\eta\). Et
\(AFN(\Sigma, \eta)\) l'ensemble des automates finis non déterministes de
valeur de transition dans \(\Sigma\) et de valeur d'état dans \(\eta\). On
écrira \(L(M)\) pour désigner le langage qu'il reconnait. Un automate est un
tuple qu'on peut écrire de cette forme \(M = (Q, I, F, \delta)\) avec~:

\begin{align*}
    Q & \subseteq \eta \quad \text{L'ensemble des états qui constituent
    l'automate}                                                                \\
    I & \subseteq Q \quad \text{L'ensemble des états initiaux}          \\
    F & \subseteq Q \quad \text{L'ensemble des états finaux}            \\
    \delta:~ & Q \times \Sigma \to 2^Q \quad \text{La fonction de transition}
\end{align*}

Un automate peut se représenter à l'aide d'un graphe orienté, valué,
particulier. Par exemple si on veut représenter \(M = (\{q_1, q_2, q_3, q_4,
q_5\}, \{q_1\},\{q_2, q_3\}, \delta)\) avec \(M \in AFN(\Sigma, \eta)\),
\(\Sigma = \{0, 1\}\), \(\eta = \{q_1, q_2, q_3, q_4, q_5\}\) et \(\delta\)
défini comme ceci~:

\begin{align*}
    \delta(q_1, 0) & = \{q_2, q_4\} & \delta(q_3, 1) & = \{q_4\} \\
    \delta(q_1, 1) & = \varnothing  & \delta(q_4, 0) & = \{q_5\} \\
    \delta(q_2, 0) & = \varnothing  & \delta(q_4, 1) & = \{q_3\} \\
    \delta(q_2, 1) & = \varnothing  & \delta(q_5, 0) & = \{q_4\} \\
    \delta(q_3, 0) & = \{q_3\}      & \delta(q_5, 1) & = \{q_5\} \\
\end{align*}

\begin{figure}[H]
    \centering
    \captionsetup{type=figure,justification=centering}
    \begin{tikzpicture}
        \tikzset{
            ->,
            >=stealth',
            node distance=3cm,
            every state/.style={thick},
            initial text=$ $,
        }
        \node[state, initial] (q1) {$q_1$};
        \node[state, accepting, right of=q1] (q2) {$q_2$};
        \node[state, above of=q2] (q4) {$q_4$};
        \node[state, accepting, right of=q2] (q3) {$q_3$};
        \node[state, right of=q4] (q5) {$q_5$};

        \draw   (q1) edge[above] node{0} (q4)
        (q1) edge[below] node{0} (q2)
        (q4) edge[bend right, below] node{1} (q3)
        (q4) edge[above] node{0} (q5)
        (q3) edge[above] node{1} (q4)
        (q3) edge[loop right] node{0} (q3)
        (q5) edge[bend right, above] node{0} (q4)
        (q5) edge[loop right] node{1} (q5);
    \end{tikzpicture}
    \caption{
        Exemple de représentation graphique d'un automate.
    }\label{fig:automata}
\end{figure}

Dans la Figure~\ref{fig:automata}, on peut voir que les états initiaux (dans
cet automate n'y a qu'un seul initial~; \(q_1\)) ont une petite flèche qui
pointe sur eux et que les états finaux ont un double contour. Et que les
transitions sont symbolisées par des flèches entre les états et que ces flèches
sont labellisées.

% \vphantom{}

% On parlera de l'inverse de l'automate \(M\) noté \(\overleftarrow{M}\) qui peut
% être défini de cette façon~:

% \begin{gather*}
%     \overleftarrow{M} = (Q, F, I, \delta') \quad \text{avec} \\
%     M = (Q, I, F, \delta) \notag \\
%     \forall (p, q) \in Q^2 ~|~ q \in \delta(p, a) \Rightarrow p \in \delta'(q, a) \notag
% \end{gather*}

% Donc si on veut représenter l'inverse de l'automate représenté dans la
% Figure~\ref{fig:automata}, ça nous donnerait ceci~:

% \begin{figure}[H]
%     \centering
%     \captionsetup{type=figure,justification=centering}
%     \begin{tikzpicture}
%         \tikzset{
%             ->,
%             >=stealth',
%             node distance=3cm,
%             every state/.style={thick},
%             initial text=$ $,
%         }
%         \node[state, accepting] (q1) {$q_1$};
%         \node[state, initial below, right of=q1] (q2) {$q_2$};
%         \node[state, above of=q2] (q4) {$q_4$};
%         \node[state, initial below, right of=q2] (q3) {$q_3$};
%         \node[state, right of=q4] (q5) {$q_5$};

%         \draw   (q4) edge[above] node{0} (q1)
%         (q2) edge[below] node{0} (q1)
%         (q3) edge[bend left, below] node{1} (q4)
%         (q5) edge[above] node{0} (q4)
%         (q4) edge[above] node{1} (q3)
%         (q3) edge[loop right] node{0} (q3)
%         (q4) edge[bend left, above] node{0} (q5)
%         (q5) edge[loop right] node{1} (q5);
%     \end{tikzpicture}
%     \caption{
%         Exemple de représentation graphique de l'inverse de l'automate de la
%         Figure~\ref{fig:automata}.
%     }\label{fig:automata_invserse}
% \end{figure}

% On voit bien que l'apparence de l'automate ne change pas les transitions sont
% juste inversées et les états initiaux sont devenus finaux et inversement.

\vphantom{}

On peut aussi étendre la fonction de transition \(\delta\) de manière qu'elle
ait comme signature~:

\[
    \delta: Q \times \Sigma^* \to 2^Q
\]

En la définissant récursivement de telle sorte~:

\begin{align*}
    \delta(q, \varepsilon) & = \{q\}                                                                       \\
    \delta(q, a \cdot w)   & = \bigcup_{q' \in \delta(q, a)} \delta(q', w) \quad \text{avec}~ a \in \Sigma
\end{align*}

\begin{example}
    Voici donc quelques exemples si on prend l'automate utilisé pour la
    représentation graphique (Figure~\ref{fig:automata})~:

    \begin{align*}
        \delta(q_1, 00)           & = \{q_5\}                                \\
        \delta(q_1, 11)           & = \varnothing                            \\
        \delta(q_1, \varepsilon)  & = \{q_1\}                                \\
        \delta(q_1, 00 \cdot 1^n) & = \{q_5\} \text{ avec } n \in \mathbb{N}
    \end{align*}
\end{example}

\vphantom{}

\begin{definition}
    Un automate est dit \textit{standard} quand il ne possède qu'un seul état
    initial non ré-entrant, aussi défini comme ceci~:

    \begin{gather*}
        M = (Q, \{i\}, F, \delta) \quad \text{avec} \\
        \forall p \in Q, \forall a \in \Sigma ~|~ i \notin \delta(p, a) \notag \\
        M \in AFN(\Sigma, \eta) \notag
    \end{gather*}
\end{definition}

\begin{definition}
    Un automate est \textit{homogène} lorsque, pour tous les états, les
    transitions allant vers cet état ont la même valeur. En d'autres termes,
    quand il respecte cette propriété~:

    \begin{gather*}
        M = (Q, I, F, \delta) \quad \text{avec} \\
        \forall (p, q, r) \in Q^3, \exists (a, b) \in \Sigma^2 ~|~ q \in \delta(p, a) \land q \in \delta(r, b) \Longrightarrow a = b \notag \\
        M \in AFN(\Sigma, \eta) \notag
    \end{gather*}
\end{definition}

\begin{definition}
    Un automate est qualifié d'\textit{accessible} lorsqu'en partant des
    initiaux, on peut arriver sur tous les états qui le composent. C'est-à-dire
    qu'il valide cette condition~:

    \begin{gather*}
        M = (Q, I, F, \delta) \quad \text{avec} \\
        \forall p \in Q, \exists w \in \Sigma^* ~|~ p \in \bigcup_{i \in I} \delta(i, w) \notag \\
        M \in AFN(\Sigma, \eta) \notag
    \end{gather*}
\end{definition}

\begin{definition}
    Un automate est considéré comme \textit{coaccessible} dès que, de tous les
    états, on peut arriver à un état final. Ceci veut dire qu'il atteste de
    cette particularité~:

    \begin{gather*}
        M = (Q, I, F, \delta) \quad \text{avec} \\
        \forall p \in Q, \exists w \in \Sigma^* ~|~ F \cap \delta(p, w) \neq \varnothing \notag \\
        M \in AFN(\Sigma, \eta) \notag
    \end{gather*}
\end{definition}

\begin{definition}
    Un automate est dit \textit{déterministe} quand tous ses états vont au
    maximum à un état par symbole et que l'automate ne possède qu'un seul état
    initial. Autrement dit qu'il valide cette propriété~:

    \begin{gather*}
        M = (Q, I, F, \delta) \quad \text{avec} \\
        |I| = 1 \land \forall q \in Q, \forall a \in \Sigma, | \delta(q, a) | \leq 1\\
        M \in AFN(\Sigma, \eta) \notag
    \end{gather*}

    On parlera de \textit{déterministe complet} lorsque tous ses états vont sur
    un état par symbole. C'est-à-dire qu'il respecte cette condition~:

    \begin{gather*}
        M = (Q, I, F, \delta) \quad \text{avec} \\
        |I| = 1 \land \forall q \in Q, \forall a \in \Sigma, | \delta(q, a) | = 1\\
        M \in AFN(\Sigma, \eta) \notag
    \end{gather*}
\end{definition}

\begin{example}
    Donc, l'automate représenté sur la Figure~\ref{fig:automata} est standard,
    non homogène, accessible et coaccessible. Car il possède bien un unique
    état initial (\(q_1\)), mais \(q_3\), \(q_4\) et \(q_5\) ne respecte pas la
    propriété pour être homogène, parce qu'ils ont des transitions allant vers
    eux avec des valeurs différentes. De plus, tous ses états sont accessibles
    depuis l'état initial. Et son inverse est, lui-même aussi, accessible et il
    n'est pas déterministe.
\end{example}

\begin{definition}
    Nous parlerons de sous-automate pour parler d'une \og{}région\fg{} d'un
    automate. \(N\) est un sous automate de \(M\) qu'on notera \(N \subseteq
    M\), s'il vérifie cette propriété~:

    \begin{gather*}
        N = (Q', I', F', \delta') \quad \text{avec} \\
        Q' \subseteq Q \land I' \subseteq Q' \land F' \subseteq Q' \\
        \delta' \text{ est une restriction de } \delta, \delta': Q' \to 2^{Q'}\\
        M = (Q, I, F, \delta) \land (M, N) \in (AFN(\Sigma, \eta))^2 \notag
    \end{gather*}
\end{definition}

Les automates pouvant être représentés à l'aide de graphes, on peut étendre les
propriétés sur les graphes aux automates. Par exemple, on pourra parler des
composantes fortement connexes d'un automate. Autrement dit, en partant de
n'importe quel état, on peut arriver à tous les autres états. Ainsi, ça veut
dire qu'un automate fortement connexe vérifierait ceci~:

\begin{gather*}
    M = (Q, I, F, \delta) \quad \text{avec} \\
    \forall (p, q) \in Q^2, \exists w \in \Sigma^* ~|~ q \in \delta(p, w) \notag \\
    M \in AFN(\Sigma, \eta) \notag
\end{gather*}

\begin{definition}
    Une autre notion qui est présente sur les graphes que nous allons adapter
    sur les automates est la notion de \textit{hamac}. Nous dirons qu'un
    automate est un \textit{hamac} lorsqu'il est standard, accessible et
    coaccessible (nous gardons le nom \textit{hamac} pour une raison de
    compréhension). Ceci veut dire qu'il peut être décrit comme ceci~:

    \begin{gather*}
        M = (Q, I, F, \delta) \quad \text{avec} \\
        standard(M) \land accessible(M) \land coaccessible(M) \\
        M \in AFN(\Sigma, \eta) \notag
    \end{gather*}
\end{definition}

\begin{definition}
    Une autre idée empruntée au graphe est la notion d'\textit{orbite}. Nous
    dirons qu'un sous-automate est une \textit{orbite}, si pour tout couple
    d'état \(i\) et \(t\), il existe un mot non vide permettant d'aller de
    \(i\) à \(t\). Aussi défini comme ceci~:

    \begin{gather*}
        \mathcal{O} = (Q, I, F, \delta) \quad \text{avec} \\
        \forall (p, q) \in Q^2, \exists w \in \Sigma^* \setminus \{\varepsilon\} ~|~ q \in \Sigma(p, w) \\
        \mathcal{O} \subseteq M \land M \in AFN(\Sigma, \eta)
    \end{gather*}

    \noindent Dans la même idée, nous parlerons d'\textit{orbite maximale}
    lorsque l'orbite n'est incluse dans aucune orbite différente. En d'autres
    termes, que l'orbite est fortement connexe sans prendre les chemins
    triviaux
    (mot vide).
\end{definition}

\begin{definition}
    Nous noterons \(In(\mathcal{O})\) et \(Out(\mathcal{O})\) respectivement
    l'ensemble des portes d'entrée et l'ensemble des portes de sortie de
    l'orbite \(\mathcal{O}\). Qui sont définies de cette façon~:

    \begin{gather*}
        In(\mathcal{O}) = \{p \in Q' ~|~ \exists a \in \Sigma, \exists q \in Q
        \setminus Q', p \in \delta(q, a)\} \cup I'\\
        Out(\mathcal{O}) = \{p \in Q' ~|~ \exists a \in \Sigma, \exists q \in Q
        \setminus Q', q \in \delta(p, a)\} \cup F'\\
        \text{avec} \\
        \mathcal{O} = (Q', I', F', \delta') \land M = (Q, I, F, \delta) \\
        \mathcal{O} \subseteq M \land M \in AFN(\Sigma, \eta)
    \end{gather*}
\end{definition}

\begin{definition}
    Avec ceci, on peut définir ce qu'est une \textit{orbite stable}. Une orbite
    est dite \textit{stable} quand pour toutes les sorties, il existe une
    transition vers toutes les entrées. C'est-à-dire que l'orbite vérifie
    ceci~:

    \begin{gather*}
        \forall q \in Out(\mathcal{O}), \exists a \in \Sigma ~|~ \delta(q, a) \cap In(\mathcal{O}) \neq \varnothing \\
        \text{avec} \\
        \mathcal{O} = (Q, I, F, \delta) \subseteq M \land M \in AFN(\Sigma, \eta)
    \end{gather*}

    \noindent On la qualifiera même de \textit{fortement stable} lorsqu'en
    supprimant toutes les transitions de portes des sorties vers les portes
    d'entrée, les orbites maximales de l'orbite sont stables et
    \textit{fortement stables}.
\end{definition}

\begin{definition}
    De même, on dira qu'une orbite est \textit{transversale} si toutes les
    entrées viennent des mêmes états et que toutes les sorties vont aux mêmes
    états. Autrement dit, que l'orbite valide cette propriété~:

    \begin{gather*}
        \forall (p, q) \in (Out(\mathcal{O}))^2, (\bigcup_{a \in \Sigma} \delta(p, a)) \cap Q \setminus Q' = (\bigcup_{a \in \Sigma} \delta(q, a)) \cap Q \setminus Q' \\
        \forall (p, q) \in (In(\mathcal{O}))^2, \{r \in Q \setminus Q' ~|~ \exists a \in \Sigma, p \in \delta(r, a)\} = \{r \in Q \setminus Q' ~|~ \exists a \in \Sigma, q \in \delta(r, a)\} \\
        \text{avec} \\
        \mathcal{O} = (Q', I', F', \delta) \subseteq M = (Q, I, F, \delta) \land M \in AFN(\Sigma, \eta)
    \end{gather*}

    \noindent Elle sera même \textit{fortement transversale} lorsqu'en
    supprimant toutes les transitions de portes des sorties vers les portes
    d'entrée, les orbites maximales de l'orbite sont transversales et
    \textit{fortement transversales}.
\end{definition}

\subsection{Fonction sur les automates}

Une des fonctions sur les automates est \(accept\) qui vérifie si le mot est
reconnu par l'automate. C'est-à-dire que si on prend le chemin décrit par le
mot donné en argument, on arrive sur un ou plusieurs états finaux. Elle a alors
pour signature~:

\[
    accept: AFN(\Sigma, \eta) \times \Sigma^* \to \mathbb{B}
\]

Elle peut être définie simplement comme ceci~:

\begin{align*}
    accept(M, w) = (\bigcup_{p \in I} \delta(p, w)) \cap F \neq \varnothing
\end{align*}

Une autre fonction sur les automates est \(homogenized\) qui renvoie l'automate
homogène qui reconnait le même langage que l'automate donné. Elle a ainsi comme
signature~:

\[
    homogenized: AFN(\Sigma, \eta) \to AFN(\Sigma, (\Sigma \cup \{\varepsilon\} \times \eta))
\]

Elle peut être définie de cette façon~:

\begin{gather*}
    homogenized(M) = N \quad \text{avec}                                                          \\
    \forall (p, q) \in Q^2, \exists a \in \Sigma ~|~ p \in \delta(q, a) \Rightarrow (a, p) \in Q' \\
    \forall (p, q) \in Q^2, \forall a \in \Sigma ~|~ p \notin \delta(q, a)
    \Rightarrow (\varepsilon, p) \in Q' \\
    \forall p \in I, \forall a \in \Sigma \cup \{\varepsilon\} ~|~ (a, p) \in Q' \Rightarrow (a, p) \in I' \\
    \forall p \in F, \forall a \in \Sigma \cup \{\varepsilon\} ~|~ (a, p) \in Q' \Rightarrow (a, p) \in F' \\
    \forall (p, q) \in Q, \exists a \in \Sigma, \forall b \in \Sigma \cup \{\varepsilon\} ~|~ (b, q) \in Q', p \in \delta(q, a) \Rightarrow (p, a) \in \delta'((b, q), a) \\
    M = (Q, I, F, \delta) \in AFN(\Sigma, \eta) \\
    N = (Q', I', F', \delta') \in AFN(\Sigma, (\Sigma \cup \{\varepsilon\}
    \times \eta))
\end{gather*}

On remarque bien que par construction l'automate résultant est homogène, parce
que les états sont devenus un couple entre leur valeur et leur transition
entrante. Ce qui fait que toutes les transitions vers l'état \((a, p)\) ont
tous pour valeur \(a\).

\subsection{Conclusion}

Comme nous venons de voir, les automates sont des outils pour reconnaitre des
mots d'un langage. L'une de leurs grandes forces est leur simplicité. Toutes
les opérations sur les automates peuvent donc être automatisées. Ce qui fait
que cet objet est très intéressant dans le monde de l'informatique. En
revanche, l'un de ses points faibles est que pour nous humain, il est difficile
de représenter un automate autrement que par une représentation graphique.
Contrairement aux expressions régulières. Il serait alors intéressant de
pouvoir convertir une expression régulière en automate. On pourrait se poser la
question \og{}est-ce-que c'est toujours possible de convertir une expression
régulière en automate\fg{}, la réponse est oui, car selon le théorème de
Kleene~\cite{Kleene1951RepresentationOE}, toute expression régulière peut être
représentée par un automate fini. Nous allons voir un algorithme pour faire
cette conversion dans la prochaine section.


\section{Les automates de Glushkov}\label{sec:glushkov}

Le terme `automates de Glushkov' est un abus de langage, faisant référence aux
automates que l'algorithme de transformation d'expression régulière en
automate, appelé algorithme de Glushkov produit. Son nom vient de
l'informaticien soviétique \textit{Victor Glushkov} qui est son créateur.

\subsection{Définition}

Nous appliquerons cet algorithme à l'aide la fonction \(glushkov\) qui a donc
pour signature~:

\[
    glushkov: E(\Sigma) \to M(\Sigma, \mathbb{N})
\]

Et a ainsi, on peut définir cette fonction de cette façon~:

\begin{gather}
    glushkov     (E(\Sigma)) = (Q, \{0\}, F, \delta) \quad \text{avec} \\
    \begin{align*}
        Q                                             & \leftarrow \{n ~|~ n \in \mathbb{N} \land 0 \leq n < m\}                                                                                               \\
        F                                             & \leftarrow \begin{cases} \{n ~|~ (a, n) \in Last\} \cup \{0\}, & \text{si } Null = \varepsilon \\ \{n ~|~ (a, n) \in Last\} & \text{sinon} \end{cases} \\
        \forall q                    \in \delta(p, a) & ~|~ \begin{cases} (a, q) \in First, & \text{si } p = 0 \\ (a, q) \in Follow((b, p)) & \text{sinon} \end{cases}                                         \\
        (E', m)                                       & \leftarrow linearization(E(\Sigma), 1)                                                                                                                 \\
        (First, Last, Null, Follow)                   & \leftarrow flnl(E')
    \end{align*}
\end{gather}

\begin{example}
    Vu qu'un dessin vaut toujours mieux que mile mots, voici un exemple de
    l'automate résultant de la transformation de cette expression \(E(\mathbb{A}) =
    (a+b) \cdot a^* \cdot b^* \cdot (a+b)^*\).

    \begin{figure}[H]
        \centering
        \captionsetup{type=figure,justification=centering}
        \begin{tikzpicture}
            \tikzset{
                ->,
                >=stealth',
                node distance=2.25cm,
                every state/.style={thick},
                initial text=\( \),
            }
            \node[state, initial] (0) {\(0\)};
            \node[state, accepting, right of=0, below of=0] (1) {\(1\)};
            \node[state, accepting, right of=0, above of=0] (2) {\(2\)};
            \node[state, accepting, right of=2, below of=2] (3) {\(3\)};
            \node[state, accepting, right of=3] (4) {\(4\)};
            \node[state, accepting, right of=4, above of=4] (5) {\(5\)};
            \node[state, accepting, right of=5] (6) {\(6\)};

            \draw   (0) edge[below] node{a} (1)
            (0) edge[above] node{b} (2)
            (1) edge[below] node{a} (3)
            (2) edge[above] node{a} (3)
            (1) edge[bend right=1.5cm, below] node{b} (6)
            (2) edge[bend left=1.5cm, above] node{b} (6)
            (1) edge[bend right, below] node{a} (5)
            (2) edge[bend left, above] node{a} (5)
            (1) edge[bend left=2mm, below] node{b} (4)
            (2) edge[bend right=2mm, above] node{b} (4)
            (3) edge[above] node{b} (4)
            (3) edge[loop left] node{a} (3)
            (3) edge[bend left, above] node{a} (5)
            (3) edge[bend right=1.75cm, below] node{b} (6)
            (4) edge[above] node{a} (5)
            (4) edge[bend right, below] node{b} (6)
            (4) edge[loop above] node{b} (4)
            (5) edge[bend right, below] node{b} (6)
            (5) edge[loop above] node{a} (5)
            (6) edge[bend right, above] node{a} (5)
            (6) edge[loop right] node{b} (6);
        \end{tikzpicture}
        \caption{
            Exemple de représentation graphique de l'automate résultant de
            \(glushkov(E(\mathbb{A}))\).
        }\label{fig:automata_glushkov}
    \end{figure}
\end{example}

\vphantom{}

Comme on peut voir sur la figure~\ref{fig:automata_glushkov} les automates
produits sont bien souvent gros et peuvent être dur à comprendre, mais une
machine peut gérer ça très simplement. Outre sa taille, on peut remarquer qu'il
y a des propriétés intéressant sur cette automate. C'est ce que l'on va étudier
maintenant.

\subsubsection*{Propriétés~:}

Nous verrons ici plusieurs propriétés sur les automates de Glushkov, mais nous
n'en ferons pas la preuve, nous en donnerons une justification, mais pas une
réelle preuve.

\vphantom{}

\begin{itemize}[label=\textbullet]
    \item La première propriété plutôt évidente, est que les automates de Glushkov sont
          \textit{standards}, car par constructions, il ne peut avoir qu'un seul état
          initial (0).

          \vphantom{}

    \item La deuxième est que l'automate à \(n + 1\) avec \(n\) le nombre de symboles de
          l'expression régulière. Le \(+ 1\) vient dû fait que nous ajoutons un état
          \(0\) qui a des transitions vers les \(First\).

          \vphantom{}

    \item La troisième propriété un peu moins flagrante est que les automates de Glushkov
          sont accessibles et coaccessibles. C'est dû au fait que chaque symbole dans
          l'expression régulière est accessible et coaccessible et que cette propriété ne
          se perd pas lors de la transformation.

          \vphantom{}

    \item La dernière propriété est que l'automate de Glushkov est homogène. Cela résulte
          de sa construction, car pour qu'un état aille sur un autre état, il faut qu'il
          ait dans ses \textit{Follow} \((a, n)\) avec \(a\) le symbole de la transition
          et \(n\) la valeur de l'état. Et étant donné que pour chaque couple \((b, m)\)
          il ne peut n'avoir que ce couple avec comme seconde valeur \(m\) alors la
          transition vers cet état sera toujours la même.
\end{itemize}

\subsection{Conclusion}

L'algorithme de Glushkov est très puissants, car permet de convertir une
expression régulière en automate ce qui fait qu'on gagne les avantages des deux
structures. Avec les expressions régulières, on peut simplement décrire un
langage et avec les automates, on peut simplement savoir si un mot est reconnu.
Il est très utilisé en \textit{informatique}, parce que pour les humains, il
est plus simple de décrire un langage avec une expression régulière. Et les
machine comprennent très facilement les automates. Ce qui fait qu'il est
possible de faire des \textit{programmes informatiques} qui reconnait un
langage et exécute des tâches à chaque mot.


\section{Conclusion}

Ce document a fourni une définition formelle des concepts clefs utilisés dans
la théorie des langages et les automates. Nous avons d'abord introduit les
notions de mots, de langages et les opérations fondamentales qui leur sont
associées. Ensuite, nous avons exploré les expressions régulières, leurs
définitions et les fonctions qui peuvent être appliquées sur elles. Par la
suite, nous avons étudié les automates, en particulier ceux sans transitions
\(\varepsilon\), et les fonctions qui leur sont appliquées. Enfin, nous avons
abordé les automates de Glushkov, décrivant leur construction et leurs
propriétés.

\vphantom{}

Bien que nous n'ayons couvert qu'une partie des concepts et des opérations
possibles, cette introduction vise à fournir une base solide pour comprendre et
utiliser ces outils puissants. Pour approfondir vos connaissances, nous vous
encourageons à consulter des ressources supplémentaires en théorie des langages
et des automates.

\printbibliography[title=Bibliographie]

\end{document}
